\chapter{Auslegung der Stromversorgung}

\section{Servomotor}
Stromverbrauch des Servomotors ist schwer zu ermitteln, da kein dauerbetrieb möglich ist und die Belastu ng nur schwer nachzustellen ist.,
darum suche nach Quelle. Quelle gefunden \cite{website-servo}
\url{http://www.flyheli.de/rxversorgung.htm}
Im Artikel ähnlich starker Servomotor, aber anderer Hersteller.
laut Artikel immer gleicher Stromverbrauch, allerdings in Intervallen, bei dauerbelastungdauerhafter Stromfluss.
Bei ähnlichem Modell 1,2A



Daten Servo:

%%Betriebsspannung: 4,8-6,0V
%%Stellzeit(60°): 0,13s (4,8V) / 0,16s (6,0V)
%%Stellmoment: 92Ncm (4,8V) / 78Ncm (6,0V)
%%Gewicht: 66,4g
%%Abmessungen: 40,6x20x37,5mm

%Stormverbrauch Servomotor:

%http://www.flyheli.de/rxversorgung.htm


\section{Pandaboard ES}

Keine angabe zum Stromverbrauch aufseiten des Herstellers.
Empfohlendes Netzteil mit 4A, aber betrieb auch an USB möglich (mit Y-Kabel) also max 2x500mA (5W).
\url{http://omappedia.org/wiki/PandaBoard_FAQ#What_are_the_specs_of_the_Power_supply_I_should_use_with_a_PandaBoard.3F} \cite{website-panda-supply}

daten für das ``normale" Pandaboard:
\url{http://omappedia.org/wiki/Panda_Test_Data} \cite{website-panda-power}
Der maximal gemessene Stromverbrauch des board nach [quelle einfügen] liegt mit 100 Prozent CPU, wlan an, und HDMI an. bei 800mA.
Der Verbrauchdes Pandaboard ES dürfte durch de Schnelleren Prozessor minimal darüber liegen. Sodas hier ein Stromverbrauch von 1A veranschalgt wird.

\section{Led Beleuchtung}
Datenblatt:
\url{http://www.adafruit.com/datasheets/WS2812.pdf} \cite{ds-WS2812}
20mA pro led,also 3x20mA
Insgesammt werden 2x front 2x heck 4x Blinker und 1x RC-Mode LEDs verbaut also 9Leds = 9*60mA = 540mA

\section{Microkontroller}
Der maximale Stromverbrauch des AVRs liegt laut Datenblatt\cite{ds-at90can} bei 200mA (Wenn IO-Pins belastet werden),
der uC selber braucht jedoch  bei 5V und 16MHz nur 29mA

\section{Sharp Sensoren}
Stromverbrauch laut Datenbaltt max 50mA, es werden 2 Sensoren vorgesehen also 100mA.\cite{ds-sharp-GP2D120}

\section{Sonstige Peripherie}
Da der Stromverbrauch der restliche Komponenten minimal ist werden hier pauschal 100mA veranschalgt.

\section{Gesammtstomverbrauch}
1200mA + 1000mA + 540mA +200mA + 100mA +100mA = 3140mA

\section{Auswahl des Reglers}