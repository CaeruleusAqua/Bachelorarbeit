
\chapter{Konzept}

\section{Grundlegender Aufbau}

Die Treiberplatine ist der zentrale Punkt für das Einsammeln aller Messwerte und die Ansteuerung der Aktorik. Dabei übernimmt sie sowohl die Energieversorgung der Komponenten als auch
die Kommunikation mit der darüber liegenden Recheneinheit. Herzstück der Platine ist dabei ein Mikrocontroller, an welchen über diverse Schnittstellen die Aktorik bzw. Sensorik
angeschlossen ist. Die Platine selber kommuniziert über USB mit der Recheneinheit und stellt dieser eine Schnittstelle zum Auslesen der Messwerte und Einstellen der Stellgrößen für die Aktorik
zur Verfügung. Weitere Aufgaben der Platine sind die Überwachung von Zuständen wie z.B. der Akkuspannung und dem Motorstrom. Eine Übersicht über die Sensorik bzw. Aktorik und ihre Anbindung ist in 
Abbildung [\ref{fig:konzept}] zu sehen.

\begin{figure}[H]
\centering
\includegraphics[page=1,width=.8\textwidth]{graph/concept.pdf}\\
\caption{Konzept}
\label{fig:konzept}
\end{figure}


Die fertig entwickelte Platine soll als Prototyp gefertigt werden, deshalb wird darauf geachtet, dass alle verwendeten Bauteile mit Hand bestückt werden können.
Die Größe aller SMD Bausteine wird aufgrund des vorhandenen Sortiments auf 0805 beschränkt.


\section{Der Mikrocontroller}
Als Mikrocontroller wird ein ``Atmel At90can12''\cite{ds-at90can} verwendet, da dieser vor Ort vorhanden ist und die nötigen Funktionen wie PWM-Kanäle und einen Analog-/Digital-Wandler mitbringt und 
über die nötige Leistung und Speicherkapazität verfügt (128KB). Der Mikrocontroller kann durch ein einzelnes 5V-Netz mit Strom versorgt werden.




\section{Motortreiber}
Da der Motortreiber eine hohe Belastbarkeit von mindestens 20 Ampere aufweisen soll und
vollintegrierte Motortreiber in dieser Leistungsklasse praktisch nicht zu bekommen sind, wird der Motortreiber als Vierquadrantensteller mit diskreten Mosfets ausgeführt.

\begin{minipage}{0.9\textwidth}
\textbf{Vierquadrantensteller:\\}
\emph{``Ein Vierquadrantensteller besteht aus einer elektronischen H-Brückenschaltung aus vier Halbleiterschaltern, meist aus Transistoren, 
welche eine Gleichspannung in eine Wechselspannung variabler Frequenz und variabler Pulsbreite umwandeln kann. Vierquadrantensteller 
in der Energietechnik können auch Wechselspannungen unterschiedlicher Frequenzen in beiden Richtungen ineinander umwandeln.''}\cite{vierquadrantensteller2}
\end{minipage}

\begin{figure}[H]
\centering
\includegraphics[width=.8\textwidth]{Vierquadrantensteller.png}\\
\caption{Vierquadrantensteller \cite{vierquadrantensteller}}%
\label{fig:Vierquadrantensteller2}
\end{figure}

\section{Spannungsversorgung}
Da der Mikrocontroller und alle vorhandenen Sensoren eine Betriebs\-spannung von 5 Volt aufweisen, wird auf ein anderes Spannungsnetz verzichtet.
Da die Platine somit nur mit einer einzigen Spannungsversorgung ausgestattet ist, werden alle weiteren Bauteile so ausgewählt, dass sie 
an einem 5V-Netz betrieben werden können.

\section{Anschluss an das Pandaboard}
Da die verwendeten Pandaboards bereits über GPIO-Anschlüsse verfügen, wird auf eine USB Anbindung verzichtet, stattdessen wird die Platine direkt an die SPI-Anschlüsse
eines der Pandaboards angeschlossen. Da die Verbindung mechanisch stabil sein muss und trotzdem einfach zu entfernen sein soll, wird hier eine RJ45-Buchse verwendet.
Diese stellt insgesamt 8 Leitungen zur Verfügung. Um eine größere Flexibilität zu erreichen wird durch diese zusätzlich die UART geleitet.
Die einzelnen Pins sind wie folgt belegt:

\begin{table}[H]
  \centering
  \begin{tabularx}{\textwidth}{|l|X|}
    \hline
     Pin & Funktion [\SI{}{\V}]  \\ \hline \hline
     1 & RX \\ \hline
     2 & MISO \\ \hline
     3 & TX \\ \hline
     4 & MOSI \\ \hline
     5 & SCK \\ \hline
     6 & 5 Volt \\ \hline
     7 & Reset (AVR) \\ \hline
     8 & GND \\ \hline
  \end{tabularx}
  \caption{Belegung der RJ45 Buchse}%
  \label{tab:rj45}
\end{table}

\begin{minipage}{0.9\textwidth}
\textbf{Hinweis:\\}
\emph{Da die Pandaboards jedoch im Laufe der Projektphase durch einen Intel NUC ersetzt wurden, welcher nicht über einen externen SPI-BUS verfügt, musste die Verbindung der Platine anders realisiert werden. 
Mit Hilfe der ebenfalls aufgeführten UART konnte mit einem UART-USB Adapter eine
Verbindung zum Intel NUC hergestellt werden. Da die Software erst im späteren Projektverlauf geschrieben wurde, nutzt sie die Anbindung über 
den UART-USB Adapter.}
\end{minipage}

