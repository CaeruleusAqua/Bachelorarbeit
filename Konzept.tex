
\chapter{Konzept}

Die Treiberplatine ist der zentrale Punkt für das Einsammeln aller Messwerte und die ansteuerung der Aktorik. Dabei übernimmt sie sowohl die Energieversorgung der Komponenten als auch
Kommunikation mit der darüber liegenden Recheneinheit. Herzstück der Platine ist dabei ein Atmel AT90CAN128 \textmu Controller an welchen über verschiedene Protokolle die Aktorik bzw. Sensorik
angeschlossen ist. Die Platine selber kommuniziert über USB mit der Recheneinheit und stellt dieser eine Schnittstelle zum auslesen der Messwerte und einstellen der Stellgrößen für die Aktorik
zur Verfügung. Weitere Aufgaben der Platine sind die Überwachung von Zuständen wie z.B. der Akkuspannung und dem Motorstromes. Eine Übersicht über die Sensorik bzw Aktorik und ihre Anbinding ist in 
Abbildung [\ref{fig:konzept}] zu sehen.

\begin{figure}[H]
\centering
\includegraphics[page=1,width=.8\textwidth]{graph/concept.pdf}\\
\caption{Konzept}
\label{fig:konzept}
\end{figure}


\section{Änderung der Anforderungen}
Nach der erfolgreichen Teilnahme am ``Carolo-Cup Junior'' im Febuar 2014, begann die Weiterentwicklung des Konzepts, während der Entwicklung kamen
einige Flaschenhalse zum vorschein. Sodas in der Projektphase die Hardwareplatform geändert werden musste. Die Rechenleistung der Pandaboards stellte sich
als unzureichend heraus und es wurden mehr Distanzsensoren gewünscht. Die Pandaboards wurden nach der Absprache mit dem Team duch ein Intel NUC vom Typ
D34010WYB erstezt. Laut Datenblatt \cite{datasheet-nuc} besitzt der NUC einen Weitbereichseingang zur Spannungsversorgung, dieser ist für 12-24 Volt zugelassen.
Sodas sich die ursprünglische Wahl der Akkus als Vorteil herausstellt. Die Spannung von 14,4 Volt der 4 Zellen LiPo Akkus passt genau in diesen Bereich, daher
kann der NUC direckt an die Akkus angeschossen werden.






















%Aufgrund der Anforderungen der elektrischen Antriebes und des Maststabes von 1:10 kann als Grundaufbau des Autos ein gängiges Ferngesteuertes Auto aus dem
%freien Handel genutzt werden. Die zur Verfügung stehenden Möglichkeiten wurden dabei im Team diskutiert und die Wahl fiel dabei auf das ``TT-01R Type E'' welches
%allen Anforderungen genügt. Tamiya ist seit langen ein im Modellbau etablierter Hersteller so das hier Ersatzteile lange verfügbar sein sollten.

%Im Bausatz des TT-01R Type E ist bereits ein Elektromotor samt Reglerelektronik enthalten. Leider verbaut Tamiya Motoren von vielen verschiedenen Herstellern,
%zu denen es leider kein Datenblatt gibt. Einige Recherchen ergaben allerdings das alle Motoren Nachbauten eines ``Mabuchi RS-540SH'' sind, von welchem ein Datenblatt
%existiert \cite{Mabuchi}

%%Stomverbauch