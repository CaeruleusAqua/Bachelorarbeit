\documentclass[11pt,twoside,a4paper]{mpreport}
\usepackage[utf8]{inputenc}
\usepackage[ngerman]{babel}
%\usepackage{a4wide}
%\usepackage[T1]{fontenc}
%\usepackage{textcomp}
%\usepackage{mathptmx}
\usepackage[scaled=0.85]{helvet}
%\usepackage{courier}           % if you really want to use Courier
%\usepackage{url}
\usepackage{cite}
\usepackage{tabularx}
\usepackage{amsmath}
\usepackage{amssymb}
\usepackage[ 
   colorlinks,        % Links ohne Umrandungen in zu wählender Farbe 
   linkcolor=black,   % Farbe interner Verweise 
   filecolor=black,   % Farbe externer Verweise 
   citecolor=black    % Farbe von Zitaten 
]{hyperref}
\usepackage{ngerman}
\usepackage{textcomp}
\usepackage{wdok-title}
\usepackage{tikz,pgfplots}
\usepackage[section]{placeins} 
\usepackage{gnuplottex}
\usepackage{here} 	
\usetikzlibrary{patterns}
\usetikzlibrary{fpu}
%%\usepackage[pdf]{pstricks}
\setlength{\parindent}{0pt}
\newcommand{\arcosh}{\mbox{arcosh }}
\pgfplotsset{compat=1.9}
\definecolor{lila}{rgb}{0,0.2,0.8}

% Examples for the definition of convenience commands
\newcommand{\package}[1]{\texttt{#1}}
\newcommand{\foreign}[1]{\emph{#1}}
\newcommand{\q}[1]{»#1«}

% Scale Courier by 0.9
% cf. <http://groups.google.de/groups?selm=yfid76obspu.fsf@triumf.ca>
%\DeclareFontFamily{T1}{pcr}{}
%\DeclareFontShape{T1}{pcr}{m}{n}{
%   <-> s*[.9]pcrr8t
%}{}

\title{Hardware/Software Codesign}
\author{Julian-Benedikt Scholle}
\supervisor{Dr. Ing. Sebastian Zug\\
  Dipl.-Inform Christoph Steup}

\begin{document}
\maketitle
\tableofcontents

\chapter{Einleitung}

Für den Hochschulwettbewerb „Carolo-Cup“ der Technischen Universität Braunschweig soll ein autonom fahrendes Fahrzeug im Maßstab von 1:10
entwickelt werden. Im Rahmen der Arbeit wird der Entwicklungsprozess der Motortreiberplatine des Fahrzeugs veranschaulicht.
Dabei werden auch die Probleme eines Projekts mit sich dynamisch ändernden Anforderungen gezeigt. 


\section{Carolo-Cup}
Der ``Carolo-Cup'' ist ein jährlicher Hochschulwettbewerb der Technischen Universität Braunschweig. Dieser bietet Studenten die Möglichkeit, sich mit der Entwicklung 
und Umsetzung von autonomen Modellfahrzeugen auseinander zu setzen \cite{website-carolo-cup}. Ziel des Wettbewerbes ist es ein möglichst kostengünstiges
und energieeffizientes Modellfahrzeug im Maßstab 1:10 zu entwickeln. Das Fahrzeug muss dabei möglichst schnell und fehlerfrei bestimmte Aufgaben
bewältigen. Die Aufgaben werden dabei in statische und dynamische Disziplinen unterteilt. 

In den statischen Disziplinen muss das Team sein Fahrzeugkonzept vor einer Jury, bestehend aus Experten aus Industrie und Forschung, verteidigen.
Dabei wird auf die Hardware- und Softwarearchitektur sowie Energiebedarf und Herstellungskosten eingegangen. Des Weiteren müssen die Lösungskonzepte
zur Bewältigung der dynamischen Disziplinen vorgestellt werden.

Die dynamischen Disziplinen bestehen aus mehreren Szenarien, dem parallelen Einparken, einem einfachen Rundkurs sowie einem Rundkurs mit Hindernissen.
Ein möglicher Rundkurs ist in \cref{fig:Rundkurs} zu sehen.

\begin{figure}[H]
\centering
\includegraphics[width=.8\textwidth]{Strecke.png}\\
\caption{Möglicher Rundkurs \cite{website-carolo-cup-regelwerk}}
\label{fig:Rundkurs}
\end{figure}

%\section{Das Auto}

%\begin{figure}[H]
%\centering
%\includegraphics[width=.8\textwidth]{Auto.jpeg}\\
%\caption{Möglicher Rundkurs \cite{website-carolo-cup-regelwerk}}
%\label{fig:Auto}
%\end{figure}


\section{Aufbau der Arbeit}
Die Arbeit gliedert sich in fünf Teile. 
In der Anforderungsanalyse wird die Ausgangssituation der Arbeit beschrieben und
analysiert, welche Funktionen benötigt werden. Im Kapitel 3, dem Konzept,
wird der Aufbau der Platine beschrieben und es werden grundlegende Komponenten
festgelegt. Das vierte Kapitel beschreibt die Umsetzung der in der Anforderungsanalyse
geforderten Funktionen. Kapitel 5 beschäftigt sich mit der Evaluierung einiger
als wichtig erachteten Komponenten und bewertet den Stromverbrauch.
Im letzten Kapitel wird das Ergebnis der Arbeit bewertet und mögliche Verbesserungen
vorgeschlagen.






\chapter{Konzept}

Die Treiberplatine ist der zentrale Punkt für das Einsammeln aller Messwerte und die ansteuerung der Aktorik. Dabei übernimmt sie sowohl die Energieversorgung der Komponenten als auch
Kommunikation mit der darüber liegenden Recheneinheit. Herzstück der Platine ist dabei ein Atmel AT90CAN128 \textmu Controller an welchen über verschiedene Protokolle die Aktorik bzw. Sensorik
angeschlossen ist. Die Platine selber kommuniziert über USB mit der Recheneinheit und stellt dieser eine Schnittstelle zum auslesen der Messwerte und einstellen der Stellgrößen für die Aktorik
zur Verfügung. Weitere Aufgaben der Platine sind die Überwachung von Zuständen wie z.B. der Akkuspannung und dem Motorstromes. Eine Übersicht über die Sensorik bzw Aktorik und ihre Anbinding ist in 
Abbildung [\ref{fig:konzept}] zu sehen.

\begin{figure}[H]
\centering
\includegraphics[page=1,width=.8\textwidth]{graph/concept.pdf}\\
\caption{Konzept}
\label{fig:konzept}
\end{figure}


\section{Änderung der Anforderungen}
Nach der erfolgreichen Teilnahme am ``Carolo-Cup Junior'' im Febuar 2014, begann die Weiterentwicklung des Konzepts, während der Entwicklung kamen
einige Flaschenhalse zum vorschein. Sodas in der Projektphase die Hardwareplatform geändert werden musste. Die Rechenleistung der Pandaboards stellte sich
als unzureichend heraus und es wurden mehr Distanzsensoren gewünscht. Die Pandaboards wurden nach der Absprache mit dem Team duch ein Intel NUC vom Typ
D34010WYB erstezt. Laut Datenblatt \cite{datasheet-nuc} besitzt der NUC einen Weitbereichseingang zur Spannungsversorgung, dieser ist für 12-24 Volt zugelassen.
Sodas sich die ursprünglische Wahl der Akkus als Vorteil herausstellt. Die Spannung von 14,4 Volt der 4 Zellen LiPo Akkus passt genau in diesen Bereich, daher
kann der NUC direckt an die Akkus angeschossen werden.






















%Aufgrund der Anforderungen der elektrischen Antriebes und des Maststabes von 1:10 kann als Grundaufbau des Autos ein gängiges Ferngesteuertes Auto aus dem
%freien Handel genutzt werden. Die zur Verfügung stehenden Möglichkeiten wurden dabei im Team diskutiert und die Wahl fiel dabei auf das ``TT-01R Type E'' welches
%allen Anforderungen genügt. Tamiya ist seit langen ein im Modellbau etablierter Hersteller so das hier Ersatzteile lange verfügbar sein sollten.

%Im Bausatz des TT-01R Type E ist bereits ein Elektromotor samt Reglerelektronik enthalten. Leider verbaut Tamiya Motoren von vielen verschiedenen Herstellern,
%zu denen es leider kein Datenblatt gibt. Einige Recherchen ergaben allerdings das alle Motoren Nachbauten eines ``Mabuchi RS-540SH'' sind, von welchem ein Datenblatt
%existiert \cite{Mabuchi}

%%Stomverbauch

\chapter{Auswahl der Komponenten}
Die Auswahl der Komponenten hängt maßgäblich von den Anforderungen und den lokalen gegebenheiten ab.


\section{Motoransteuerung}

\usnsection{Treiberbausteine}
Da die gewählten Akkus eine Spannung von 14,4V aufweisen, kann der orgiginal Motortreib erleider nicht verwendet werden.
Denn dieser benötig eine Spannung von 7,4V. Da der AVR Mikrokontroller mit 5V betrieben wird, wird ein Motorteiber benötig der
mit den 5V Pegeln arbeiten kann. In vielen Mikrocontroller Projekten und in unserem ersten Prototyp wird der L298 DUAL FULL-BRIDGE DRIVER
verwendet. Dieser ist leider auch bei der Benutzung beider Kanäle auf 4 Ampere begrenzt \cite{L298}, was beim Prototy zu einer Permanenten
überlastung des Treibers führt. Leider sind keine vollintegrierten Motortreiber mit der benötigten Belastbarkeit verfügbar.
Um die bentötigte Belastbarkeit zu erreichen wird der zur Ansteuerung benötigte Vierquadrantensteller aus diskreten Mosfets aufgebaut.

\subsubsection{Vierquadrantensteller}
Definition nach Wikipedia:\\
``Ein Vierquadrantensteller besteht aus einer elektronischen H-Brückenschaltung aus vier Halbleiterschaltern, meist aus Transistoren, 
welche eine Gleichspannung in eine Wechselspannung variabler Frequenz und variabler Pulsbreite umwandeln kann. Vierquadrantensteller 
in der Energietechnik können auch Wechselspannungen unterschiedlicher Frequenzen in beiden Richtungen ineinander umwandeln.''


\begin{figure}[H]
\centering
\includegraphics[width=.8\textwidth]{Vierquadrantensteller.png}\\
\caption{Vierquadrantensteller}%
\label{fig:Vierquadrantensteller}
\end{figure}


Auf Grund der hohen Belastbarkeit werden meißt Mosfets als Halbleiterschalter genutzt. Um die beiden oberen Mosfets (T1/T3) durchzuschalten
ist auf Grund des fehlenden Massepotentials eine Gatespannung oberhalb der Betriebsspannung nötig. Diese wird meist mittels Bootstrapping zur
Verfügung gestellt. Da das simultane Durchschalten der Übereinander liegenden Mosfets zu einem Kurzschluss führen würde, muss dies durch
eine Schutzschalung verhindert werden. Um all diese Funktionen zur verfügung zu stellen gibt es bereits fertige Mosfettreiber,
was das Schaltungsdesign enorm vereinfacht.

\subsection{Mosfettreiber}
\subsubsection{Verfügbarkeit}

Mosfettreiber gint es in vielenAusführungen, unter anderem als ``Single Channel High Side Driver``, ``Half Bridge Driver'', ``Full Bridge Driver''
und ``3 Phase Driver''. Da für den Verbauten DC-Motor eine Vollbrücke nötig ist, um den Motro in alle Richtungen zu betreiben, Werden an dieser Stelle
ausschließlich ``Full Bridge Driver'' untersucht.

Eine Tabelle auf Mikrocontroller.net\cite{FET_D_TABLE} zeigt eine Auswahl an verfügbaren Mosfettreibern. Dort sind zwei
``Full Bridge Driver'' aufgeführt, welche für dieses Projekt passend sind. Allerdings fällt die Entscheidung auf einen anderen Treiber,
dem Allegro A3941.
\subsubsection{Allegro A3941}
Der Allegro A3941 ist f+r Betriebsspannungen von 5,5V bis 50V geeignet und liegt damit in der Spezifikation des Projekts.
Des weiteren verfügt der Motor über eine integrierten 5V Regulator und kann somit ohne Spannungsregulator am Akku betrieben werden.
Über zwei Ausgänge der Treibers können diverse Fehler ausgelesen werden.


Der Treiber lässt sich in verscjiedennenModi betreiben:

\begin{figure}[H]
\centering
\includegraphics[width=.8\textwidth]{3941_1.png}\\
\caption{Slow decay, diode recirculation, high-side PWM}%
\label{fig:3941_1}
\end{figure}

Konfiguration: PWML=1, PHASE=1, SR=0 und PWM an PWMH (high-side PWM)\\
Bei aktivierten PWMH fließt der Strom durch den GHA-Mosfets über den Motor und
dann über den GLB-Mosfet. in diesem Modus wird der Motor angetrieben.
Wenn PWML deaktiviert ist zirkuliert der vom Motor induziert Strom durch GLB und durch
die interne Diode von GLA, der Motor wird dadurch gebremst.


\begin{figure}[H]
\centering
\includegraphics[width=.8\textwidth]{3941_2.png}\\
\caption{Slow decay, SR active, high-side PWM}%
\label{fig:3941_2}
\end{figure}

Konfiguration: PWML=1, PHASE=1, SR=1 und PWM an PWMH (high-side PWM)\\
Bei aktivierten PWMH fließt der Strom durch den GHA-Mosfets über den Motor und
dann über den GLB-Mosfet. in diesem Modus wird der Motor angetrieben.
Wenn PWML deaktiviert ist zirkuliert der vom Motor induziert Strom durch 
GLB und durch GLA, der Motor wird durch den niedrigeren Innenwiederstand des Mosfest 
stärker gebremst als in der Voherigen Konfiguration. Dabei ist darauf zu achten, dass
beinahe die gesamte induzierte Spannung über den beiden Mosfets (GLA/GLB) abfällt.
Was zu einer starken Hitzeentwicklung führen kann.



\begin{figure}[H]
\centering
\includegraphics[width=.8\textwidth]{3941_3.png}\\
\caption{Slow decay, SR active, low-side PWM}%
\label{fig:3941_3}
\end{figure}

Konfiguration: PWMH=1, PHASE=1, SR=1 und PWM an PWML (low-side PWM)\\
Diese Konfiguration entspricht im Grunde den beiden vorherigen, Nur das dass PWM-Signal
an den unteren Mosfets anliegt. Der SR-Pin entscheidet wieder darüber ob im ``Bremsmodus''
die internen Dioden genutzt werden (SR=0) oder nicht (SR=1)




\begin{figure}[H]
\centering
\includegraphics[width=.8\textwidth]{3941_4.png}\\
\caption{Fast decay, diode recirculation}%
\label{fig:3941_4}
\end{figure}


Konfiguration: PWMH=1, PWML=1, PHASE=1, SR=1\\
in dieser Konfiguration werden die oberen und unteren Mosfets gleich geschaltet. Im
``Bremsmodus'' führt das dazu das der induzierte Motorstrom nicht über die Mosfets
zirkulieren kann. Der Strom fließt stattdessen zurück in die Spannungsquelle, was
abhängig von der Spannungsquelle zu Schäden führen kann. Wird die Schaltung jedoch an
einem Akku betrieben ist es so möglich die Energie zu nutzen und damit den Akku
zu laden.

\begin{figure}[H]
\centering
\includegraphics[width=.8\textwidth]{3941_5.png}\\
\caption{Fast decay, SR active, full four-quadrant control}%
\label{fig:3941_5}
\end{figure}

Diese Konfiguration zeigt den Einfluss des PHASE-Pins. Liegt am PHASE-Pin 1 ein an
fließt der Strom von links nach rechts. Liegt 0 an fließt er von rechts nach links.
Mithilfe des PHASE-Pins wird also die Polung des Motors festgelegt.




\chapter{Auslegung der Stromversorgung}

\section{Servomotor}
Stromverbrauch des Servomotors ist schwer zu ermitteln, da kein dauerbetrieb möglich ist und die Belastu ng nur schwer nachzustellen ist.,
darum suche nach Quelle. Quelle gefunden \cite{website-servo}
\url{http://www.flyheli.de/rxversorgung.htm}
Im Artikel ähnlich starker Servomotor, aber anderer Hersteller.
laut Artikel immer gleicher Stromverbrauch, allerdings in Intervallen, bei dauerbelastungdauerhafter Stromfluss.
Bei ähnlichem Modell 1,2A



Daten Servo:

%%Betriebsspannung: 4,8-6,0V
%%Stellzeit(60°): 0,13s (4,8V) / 0,16s (6,0V)
%%Stellmoment: 92Ncm (4,8V) / 78Ncm (6,0V)
%%Gewicht: 66,4g
%%Abmessungen: 40,6x20x37,5mm

%Stormverbrauch Servomotor:

%http://www.flyheli.de/rxversorgung.htm


\section{Pandaboard ES}

Keine angabe zum Stromverbrauch aufseiten des Herstellers.
Empfohlendes Netzteil mit 4A, aber betrieb auch an USB möglich (mit Y-Kabel) also max 2x500mA (5W).
\url{http://omappedia.org/wiki/PandaBoard_FAQ#What_are_the_specs_of_the_Power_supply_I_should_use_with_a_PandaBoard.3F} \cite{website-panda-supply}

daten für das ``normale" Pandaboard:
\url{http://omappedia.org/wiki/Panda_Test_Data} \cite{website-panda-power}
Der maximal gemessene Stromverbrauch des board nach [quelle einfügen] liegt mit 100 Prozent CPU, wlan an, und HDMI an. bei 800mA.
Der Verbrauchdes Pandaboard ES dürfte durch de Schnelleren Prozessor minimal darüber liegen. Sodas hier ein Stromverbrauch von 1A veranschalgt wird.

\section{Led Beleuchtung}
Datenblatt:
\url{http://www.adafruit.com/datasheets/WS2812.pdf} \cite{ds-WS2812}
20mA pro led,also 3x20mA
Insgesammt werden 2x front 2x heck 4x Blinker und 1x RC-Mode LEDs verbaut also 9Leds = 9*60mA = 540mA

\section{Microkontroller}
Der maximale Stromverbrauch des AVRs liegt laut Datenblatt\cite{ds-at90can} bei 200mA (Wenn IO-Pins belastet werden),
der uC selber braucht jedoch  bei 5V und 16MHz nur 29mA

\section{Sharp Sensoren}
Stromverbrauch laut Datenbaltt max 50mA, es werden 2 Sensoren vorgesehen also 100mA.\cite{ds-sharp-GP2D120}

\section{Sonstige Peripherie}
Da der Stromverbrauch der restliche Komponenten minimal ist werden hier pauschal 100mA veranschalgt.

\section{Gesammtstomverbrauch}
1200mA + 1000mA + 540mA +200mA + 100mA +100mA = 3140mA

\section{Auswahl des Reglers}


\chapter{Änderung der Anforderungen}
Nach der erfolgreichen Teilnahme am ``Carolo-Cup Junior'' im Febuar 2014, begann die Weiterentwicklung des Konzepts, während der Entwicklung kamen
einige Flaschenhalse zum vorschein. Sodas in der Projektphase die Hardwareplatform geändert werden musste. Die Rechenleistung der Pandaboards stellte sich
als unzureichend heraus und es wurden mehr Distanzsensoren gewünscht. Die Pandaboards wurden nach der Absprache mit dem Team duch ein Intel NUC vom Typ
D34010WYB erstezt. Laut Datenblatt \cite{datasheet-nuc} besitzt der NUC einen Weitbereichseingang zur Spannungsversorgung, dieser ist für 12-24 Volt zugelassen.
Sodas sich die ursprünglische Wahl der Akkus als Vorteil herausstellt. Die Spannung von 14,4 Volt der 4 Zellen LiPo Akkus passt genau in diesen Bereich, daher
kann der NUC direckt an die Akkus angeschossen werden.
\include{Motortreiber}


\chapter{Motorstrommessung am Shunt}


\section{Problem}

An einem mit PWM angesteuertem DC-Motor soll eine Strommessung mit Hilfe eines Shuntwiederstandes
durchgeführt werden. Aufgrund der PWM Ansteuerung muss der DC-Anteil aus dem Signal herausgefiltert werden!


\section{Prinzip der Strommessung}

Die Messspannung wird über einen Shuntwiederstand zur Masse gemessen! Aufgrund nicht vorhandener Datenblätter des Motors
wird von einem expirimentel Ermittelten maximalen Strom des Motors ausgegangen. Dieser beträgt bei einer Betriebsspannung von 20V ca. 20A.
Da einen Shunt mit einer maximalen Belastbarkeit von 2 Watt eingesetzt wird, darf der maximale Spannungsabfall am Shunt 100mV nicht überschreiten.
Nach dem Ohmschen Gesetz ergibt sich dadurch ein Widerstand von $0,005 \Omega$  für den Shunt. Shuntwiederstände in der Größe sind problemlos zu bekommen.
Da es sich hier um eine Worst Case Rechnung handelt, wird der zusätzliche Widerstand des Shuntwiederstandes und der damit verringerte Strom bewusst ignoriert.

Die über den Shuntwiederstand gemessene Spannung soll über den ADC Eingang des Mikrocontrollers eingelesen werden. Vorher jedoch muss das Signal gefiltert werden, da der Strom
durch die Ansteuerung mittels der Pulsweitenmodulation nicht konstant ist!



\section{Anforderungen}
Die maximale Auflösung des Mikrocontrollers soll ausgenutzt werden. Der ADC des Mikrocontrollers arbeitet mit einer Auflösung von 10 Bit und einer 
Referenzspannung von 5V. Um die Auflösung des ADC auszunutzen muss das Signal, aufgrund unseres Spannungsabfalls verstärkt werden.

Als Anforderung ergibt sich außerdem, dass der maximale Ripple des Endsignals kleiner ist als der Quantisierungsfehler des ADC.
So ist es möglich aufeine zusätzliche digitale Filterung weitgehend zu verzichten.
Die kleinst mögliche zu erfassende Spannung des ADC beträgt $\frac{5}{2^{10}}=4,88mV$.
Diesen wert sollte der Ripple des Endsignales nicht überschreiten.
Aus einem möglichst kleinem Ripple resultiert eine möglichst hohe Filterordnung bzw. eine niedrige Grenzfrequenz.
Allerdings soll $U_{DC}$ einer Änderung des Mittelwertes, also einer Änderung des Tastverhältnisses, möglichst
schnell folgen. Diese Anforderung widerspricht der Vorherigen, so das ein Kompromiss gefunden werden muss.

\section{Bestimmung des Filtertyps}
Aufgrund des sehr niedrigen Spannungspegels am Shunt,ist es nötig das Signal zu verstärken. Da zum verstärken des Signals aktive elektronische Elemente notwendig sind,
z.B. ein Operationsverstärker, wird an dieser Stelle gleich ein aktiver Filter verwendet. Dieser gibt uns die Möglichkeit des Messignal zu verstärken und gleichzeitig zu
Filtern. Da wir als unser Signal im optimalen Fall eine Gleichspannung darstellt müssen wir die Hochfrequenten Anteile unseres Signales herausfiltern, dies geschieht 
mit Hilfe eines Tiefpasses. Es gibt im Grunde 2 übliche aktive Tiefpässe, den Tiefpass mit Mehrfachgegenkopplung und den Sallen-Key Filter. Ersterer verwendet
einen invertierenden Verstärker, dieser invertiert das Messsignal. Da der \textmu Controller jedoch nur mit positiven Spannungen umgehen kann müsste man hier mit einer 
negativen Referenzspannung Arbeiten, was den Schaltungsaufwand unnötig vergrößern würde. Der Sallen-Key Tiefpass benutzt einen nicht invertierenden Verstärker, welcher diesen
Nachteil nicht hat. So dass ab dieser Stelle ein Sallen-Key Tiefpass entwurfen wird.

%%TODO Quellen (verweise auf übliche Filtertypem)

\section{Die Filterschaltung}

wie im vorherigen Abschnitt diskutiert wird hier ein Sallen-Key Tiefpass entwurfen. Zum Entwurf der Schaltung wurde Eagle genutzt.

\begin{figure}[H]
\centering
\includegraphics[width=.8\textwidth]{filter_schaltung.png}\\
\caption{Salle-Key Tiefpass mit Shunt}%
\label{fig:fschalt}
\end{figure}



\section{Dimensionierung des Verstärkers}

In bisherigen Rechnungen wurde ein maximaler Spannungsabfall von 100mV am Shunt errechnet. Da der Messbereich des voll ADC ausgenutzt werden soll,
ist es nötig das Messsignal zu verstärken. Hierzu wir ein Nichtinvertierender Verstärker benutzt. Da der Messbereich des ADC bis 5V reicht, wird hier eine 
50 fache Verstärkung angestrebt.

Für einem Nichtinvertierenden Verstärker ergibt sich dann:

\begin{align*}
v &= 1 + \frac{R_{F3}}{R_{F4}}\\
50 &= 1 + \frac{R_{F3}}{R_{F4}}\\
49\cdot R_{F4} &= R_{F3}
\end{align*}
\\
Wobei $R_{F4} = 47 k\Omega$ und $R_{F3} = 1 k\Omega$  gewählt werden, was eine Verstärkung von 48 ergibt.

%%TODO Warum werde widerstände so gewählt

\section{Anforderungen an den Filter}

\begin{figure}[H]
\centering
\includegraphics[width=.8\textwidth]{oszi.png}\\
\caption{Spannung am Shunt + PWM}%
\label{fig:pwm+i}
\end{figure}

Da dem Messsignal wie in Abbildung \ref{fig:pwm+i} zu erkennen, die PWM Frequenz zu Grunde liegt wird sich bei der Dimensionierung des Filters einer Idee nach \cite{Alter2008} bedient, nach der die maximale Amplitude des Ripple der Grundschwingung bei einem
Tastverhältnis von 0,5 entspricht. Die Amplitude der Grundschwingung ergibt sich aus dem ersten Koeffizienten der Fourierreihe einer Rechteckschwingung.
\begin{align}
A_1 = K\cdot \frac{1}{\pi}[\sin(\pi p)-\sin(2\pi(1-\frac{p}{2}))]
\label{eq:ripple}
\end{align}
Wobei $p$ dem Tastverhältnis und $K$ der maximale Amplitude des Ursprungsingals entspricht \cite{Alter2008}. $K$ entspricht den errechneten 100mV multipliziert mit dem Verstärkungsfaktor 48, also 4,8V. Das Tastverhältnis $p$ wird zu 0,5
angenommen. Mit (\ref{eq:ripple}) ergibt sich für die Amplitude der Grundschwingung $ A_1 = K\cdot \frac{2}{\pi} = 3,056V$. $A_1$ soll auf $ < 4,88mV$ gedämpft werden.
Als Sperrfrequenz $\Omega_s $ wird hier die PWM Frequenz angesetzt. Für $H(\omega=2\pi f_{PWM})$ gilt also:

\begin{align}
H(\omega=2\pi f_{PWM}) \le \frac{4,88mV}{3,056V} \mathop{\hat{=}} 20\cdot\log(\frac{4,88mV}{3,056V})= -55,9 dB
\label{eq:daempfung}
\end{align}

Da das Projekt möglichst kostensparend durchgeführt werden soll, also auch Bauteilsparend, wird im Folgenden von den üblichen Konventionen zur dimensionierung von Filtern abgewichen.
Statt eine fixe Grenzfreqeunz festzulegen und die benötigte Filterordnung zu bestimmen, wird die Filterordnung vorgegeben und die Grenzfrequenz variiert.

\section{Filterentwurf}

\subsection{Bestimmung des Filtertyps}

Des Filtertyp muss in zweierlei Hinblick bestimmt werden. Einmal im hinblick auf die Schaltung und seinem Frequenzgang.
Im groben gibt es 2 mögliche aktive Tiefpassfilterschaltungen, den Sallen-Key Teifpass mit nicht invertierendem OPV und dem aktiven Tiefpass mit Mehrfachgegenkopplung 
(invertierender OPV). Der aktive Tiefpass mit Mehrfachgegenkopplung benötigt allerdings negative Spannungsniveaus die auf der Treiberplatine nicht zur
verfügung stehen, deshalb wird an dieser Stelle nur der Sallen-Key Teifpass betrachtet.
Was den Frequenzgang angeht gibt es viele Filtercharakteristiken, eine Auswahl an haufig verwendeten Charakteristiken wird hier verglichen.

Der \emph{Butterworth}-Filter besitzt einen maximal flachen Verlauf des Frequenzganges im Durchlassbereich und eine monoton verlaufende Dämpfung im Sperrbereich.
Leider hat der Butterworth-Filter nur eine geringe Flankensteilheit im Sperrbereich (20dB/Dekade pro Ordnung). Ein Butterworth-Filter 1. Ordnung entspricht einen  einfachen RC-Filter.

Der \emph{Tschebyscheff}-Filter hat eine höhere Flankensteilheit als der Butterworth-Filter, allerdings entsteht beim Tschebyscheff-Filter Welligkeit im Durchlassbereich,
welche mit höherer Ordnung zunimmt. Durch die Welligkeit im Duchlassbereich würde ein zusätzlicher Ripple im Signal entstehen, weshalb der Tschebyscheff-Filter nicht
für den geforderten Filter geeignet ist 

Der \emph{Bessel-Filter} hat den Vorteil einer konstanten Gruppenlaufzeit, hat dafür aber eine noch geringere Flankensteilheit als der Butterworth-Filter.
Da eine konstante Gruppenlaufzeit für den geforderten Filter nicht von Vorteil ist, da das Endsignals einer Gleichspannung entsprechen sollte, ist der Butterworth-Filter
die bessere Wahl.


\begin{figure}[H]
\centering
\begin{tikzpicture}
	\draw[->,thick] (0,0) -- (7.5,0) node[right] {$f[\text{Hz}]$};
	\draw[->,thick] (0,0) -- (0,3.3) node[above] {$a[\text{dB}]$};
	\draw (0,2.5)node[left] {$a_{\text{min}}$} (-0.1,2.5)--(2.9,2.5);
	\draw (0,1)node[left] {$a_{\text{max}}$};
	\def \bsp{(0,1)--(1,1)--(1,2.4)--(1,2.4)--(0,2.4)}
	\draw (-0.1,1)--(1,1)--(1,2.4) (1,0)node[below] {$f_g$};
	\pattern[pattern=north east lines] \bsp;
	\draw[dashed] (1,1) -- (1,-0.1);
	\def \bsd{(3,0) -- (3,2.5) -- (7,2.5) -- (7,0)}
	\pattern[pattern=north east lines] \bsd;
	\draw (3,0)node[below] {$f_s$} -- (3,2.5) -- (7,2.5);

\end{tikzpicture}
\caption{Tiefpass Toleranzfeld}%
\label{fig:analog}
\end{figure}
Für unsere Schaltung wird ein Sallen Key Tiefpass 2. Ordnung entwurfen. Die PWM-Frequenz $f_{PWM}$ beträt 3,9kHz.
Die Sperrfrequenz entspreicht der PWM Frequenz, also der Frequenz unserer Grundschwingung. $\Omega$ entspricht der mit der Grenzfreqeunz 
normierten Frequenz $\Omega=\frac{f}{f_g}$. Nach (\ref{eq:daempfung}) ergibt sich für Abbildung \ref{fig:analog}
$f_s=f_{PWM}=3,9 kHz$, $a_{min}=55,9 dB$ und $a_{max}$ wird auf 3dB festgelegt.



\subsection{Butterworth}
\subsection{Bestimmung der Grenzfreqeunz}
\begin{align}
n \ge \frac{\log{\sqrt{\frac{e^{2a_{min}}-1}{e^{2a_{max}}-1}}}}{\log{\Omega_s}}
\label{eq:butterworth}
\end{align}
Die Filterordnung nach Butterworth wird nach (\ref{eq:butterworth}) bestimmt. Umgestellt nach $\Omega_s$ ergibt sich:

\begin{align}
\Omega_s \le  \left(\frac{e^{2a_{min}}-1}{e^{2a_{max}}-1}\right)^{\frac{1}{2n}}
\end{align}



Für die Berechnung der Sperrfrequenz $\Omega_s$ müssen  $a_{min}$ und $a_{max}$ in Neper umgrechnet werden. Wobei:
\begin{align*}
1 \text{dB} =  \frac{\ln{10}}{20}\text{Np} = 0,115129255 \text{Np}   
\end{align*}

Damit ergibt sich für $a_{min}=55,9 dB\cdot \frac{\ln{10}}{20}=6,45Np$ und für  $a_{max}=3 dB\cdot \frac{\ln{10}}{20}=0,345Np$. Die Filterordnung wird auf 2 festgelegt.
\begin{align}
\Omega_s \le  \left(\frac{e^{2\cdot6,45N }-1}{e^{2\cdot 0,345Np}-1}\right)^{\frac{1}{2n}}  = 35,8
\end{align}

Die Grenzfreqeunz $f_g$ ergibt sich jetzt aus:

\begin{align}
\frac{f_s}{\Omega_s} \le \frac{3,9kHz}{35,8} = 108,9Hz
\end{align}

\subsubsection{Filterentwurf}
Im voherigen Abschnitt wurde berechnet das die Grenzfreqeunz der Filters kleiner als 108,9Hz sein muss.
Im Folgenden wird nun ein Sallen-Key Filter 2. Ordnung mit einer Grenzfrequenz von 100Hz entwurfen.
Die genaue Wahl der Grenzfreqeunz ist hier nicht relevant da die realen Bauteile nicht in  allen Größen 
verfügbar sind und daher am Schluss variiert werden müssen, wodurch sich die Grenzfrequen des Filters leicht ändert.


\subsubsection{Finaler Entwurf}



Betrachten wir das Polstellen-Nullstellendiagramm eines Butterworth Filters 2. Ordnung, wie in Abbildung [\ref{fig:filter_polnul}]


\begin{figure}[H]
\centering
\begin{tikzpicture}
	\draw[->,thick] (-3,0) -- (3,0) node[right] {$\text{Re}$};
	\draw[->,thick] (0,-3) -- (0,3) node[above] {$\text{Im}$};
	\draw[dashed,red,very thin] (-3,2) -- (3,2);
	\draw[dashed,red,very thin] (-3,-2) -- (3,-2);
	\draw[dashed,red,very thin] (2,-3) -- (2,3);
	\draw[dashed,red,very thin] (-2,-3) -- (-2,3);
	\draw[dashed,blue,very thin] (0,0) circle (2);
	\coordinate (x) at (225:2); 
	\coordinate (y) at (135:2);
	\draw[very thin] (0,0) -- (y);
	\draw[red,thick] (x) -- +(0.1,0.1)  (x) -- +(-0.1,-0.1) (x) -- +(0.1,-0.1) (x) -- +(-0.1,0.1);
	\draw[red,thick] (y) -- +(0.1,0.1)  (y) -- +(-0.1,-0.1) (y) -- +(0.1,-0.1) (y) -- +(-0.1,0.1);
	\draw (2,0)node[below] {$1$};
	\draw (-2,0)node[below] {$-1$};
	\draw (0,2)node[left] {$1$};
	\draw (0,-2)node[left] {$-1$};
	\draw (0,-2)node[left] {$-1$};
	\draw (0,0) (135:1cm) arc (135:180:1cm);
	\draw (-0.6,0.3)node {$\delta$};
\end{tikzpicture}
\caption{Polstellen-Nullstellendiagramm, Butterworth 2. Ordnung}
\label{fig:filter_polnul}
\end{figure}



Charakteristisch für den Butterworthfilter ist das sich die Polstellen auf einer Kreisbahn befinden. Auf die Grenzfreqeunz normiert hat dieser beim Butterworthfilter den Radius
eins. Bei einem Butterworth 2. Ordnung befinden sie sich genau bei $\delta=45^\circ$. Das Interessante am Polstellen-Nullstellendiagramm ist, dass sich Polfrequenz $\Omega_P$ und 
Polgüte $Q_P$ einfach ablesen lassen. Die Polfrequenz $\Omega_P$ ist der Betrag der normierten Polstelle, welcher beim Butterworth-Filter immer eins ist.
Die Polgüte ist abhängig von $\delta$ und ergibt sich zu: $Q_P=\frac{1}{2\cos{\delta}}$. Für unseren Butterworthfilter ergeben sich also $Q_P=0,707$ und $\Omega_P=1$


Betrachten wir deie Übertragungsfunktion eines Sallen-key Tiefpasses 2. Ordnung:

\begin{align*}
A(P)&=\frac{A_0}{1+\omega_g (R_2 C_2 + R_1 C_2 + R_1 C_2(1-A_0))P + \omega_g^2R_1 R_2 C_1C_2P^2}
\end{align*}

mit
\begin{align*}
A_0=1+\frac{R_6}{R_5}
\end{align*}


Die Bauteilwerte erhält man durch einen Koeffizientenvergleich mit der entnormierten
Übertragungsfunktion ($P=\frac{s}{\omega}$) eines Tiefpasses zweiter Ordnung:

\begin{align*}
A(P)&=\frac{A_0}{1+\frac{1}{\omega_g\Omega_PQ_P}s+\frac{1}{\omega_g^2\Omega_P^2}s^2}
\end{align*}

Die Auflösung des Vergleiches ist mit vielen Mathematischen umformungen verbunden, deswegen wird hier auf eine
externe Quelle verwiesen \cite[S. 102]{Krucker2000}.
Nach dem Koeffizientenvergleich ergibt sich

\begin{align*}
C_1&<\frac{C_2\cdot(1+4Q^2_P(A_0-1))}{4Q^2_P}\\
R_1&=\frac{1}{2\omega_g\Omega_PQ_P} \cdot \frac{C_2\pm\sqrt{C_2^2-4Q^2_PC_2(C_1+C2(1-A_0))}}{C_2(C_1-C_2(1-A_0))}   \\
R_2&=\frac{1}{2\omega_g\Omega_PQ_P} \cdot \frac{C_2\pm\sqrt{C_2^2-4Q^2_PC_2(C_1+C2(1-A_0))}}{C_1C_2}  \\
Q_p&=\frac{\sqrt{R_1R_2C_1C_2}}{C1(R_1+R_2)+R_1C_2(1-A_0)}\\
\Omega_p&=\frac{1}{\omega_g\sqrt{R_1R_2C_1C_2}}
\end{align*}

Dabei sind immernur die positiven, reellen Lösungen zu verwenden. Schließlich git es in der Realität keinen negativen Wiederstand, leider.


\subsubsection{Bestimmung der Bauteilwerte}


$Q_P=0,707$ und $\Omega_P=1$, $A_0=48$, $\omega_g = 100Hz$
$A_0$ ist die Gleichspannungsverstärkung, sie beschreibt den gewünchten Verstärkungsfaktor der Bereits in einem voherigen
Abschnitt mit 48 bestimmt wurde. Die Berechnungen wurden mit Hilfe eines Python-Scriptes ausgeführt, dabei wurden verschiedenne
Konfigurationen durchgerechnet. Hautpsächlich wurde dabei darauf geachtet, dass sich der Filter mit den vor Ort vorhandennen SMD-Bauteilen
aufgebaut werden kann.

In den Berechnungen viel auf, dass bei steigender größe der Kondensatoren die Größe der Wiederstände sinkt. Da Wiederstände auch in großen Größen vorhanden waren,
Wurde für den frei wählbaren $C_2$ ein kleiner Wert von 82nF gewählt.

\begin{align*}
C_1&<\frac{C_2\cdot(1+4\cdot0.707^2_P(48-1))}{4\cdot0.707^2}\\
C_1&<3.90\mu F
\end{align*}

$C_1$ soll nur kleiner sein als 3.90\textmu F und wird ebenfalls auf 82nF gesetzt.

\begin{align*}
R_1&=\frac{1}{2\cdot100Hz\cdot0,707} \cdot \frac{82nF\pm\sqrt{82nF^2-4\cdot0.707^2\cdot82nF(82nF+82nF(1-48))}}{82nF(82nF-82nF(1-48))}\\
R_1&=[-3176\Omega,2579\Omega]
\end{align*}


\begin{align*}
R_2&=\frac{1}{2\cdot100Hz\cdot0,707} \cdot \frac{82nF\pm\sqrt{82nF^2-4\cdot0.707^2\cdot82nF(82nF+82nF(1-48))}}{82nF^2}\\
R_2&=[146079\Omega,-118626\Omega]
\end{align*}

Da nur positive Werte genutzt werden, ergeben sich die Bauteilwerte nun zu:
\begin{align*}
C_1&=82nF\\
C_2&=82nF\\
R_1&=2579\Omega\\
R_2&=146079\Omega
\end{align*}

In der Folgenden Abbildung ist das Ergebniss der Simulation zu sehen. An der Abbildung leider nicht gut zu erkennen,
liegt der -3dB Punkt genau bei 100Hz. Die Frequenzachse des Diagrammes geht genau bis 3,9kHz. 
Es ist eine Verstärkung von 48 des Ursprungssignals gewünscht. Diese Verstärkung wird mit 33,6 dB bei 10Hz, erreicht.
\begin{align*}
20\cdot\log{48}=-33,6dB
\end{align*}

Bei 3,9kHz erreicht der Filter eine Dämpfung von -30,1dB zusammen mit der Verstärkng von 33,6dB unseres Eingangssignals,
wird das Bereits verstärkte Signal also um 63,7 dB gedämpft. Gefordert waren hier 55,9dB, so das der Filter den gerforderten wert übersteigt, wass an der niedrigeren Grenzfreqeunz von 100Hz statt 108,9Hz liegt.

%%Verstärkung: 33,6dB gewünscht:
%%20*log(1/48)=-33,6

%%dämpfung bei 3,9Khz = 30,7dB
%% 33,6+30,1 =63,7 gewüncht: 55,9
%% Grenzfreqeunz = 100Hz (-3db)
\begin{figure}[H]
\centering
\begin{gnuplot}[terminal=pdf]
  set nokey 
  set xrange [10:3900]
  set xlabel 'Frequenz in [Hz]'
  set ylabel 'Verstärkung in [dB]'
  set logscale x 10
  plot 'Simulation/Filter_original_frequenzgang.csv' with line, 'Simulation/Filter_real_frequenzgang.csv' with line
\end{gnuplot}
\caption{Frequenzgang des berechneten Filters}
\label{plott:filter_freq}
\end{figure}

Leider kann ein solcher Filter nur mit erheblichen Aufwändungen gebaut werden, da es keine fertigen Wiederstände in den Größen $2579\Omega\\$ und $146079\Omega$ gibt. Da jedoch alle Wiederstände der E12 Reihe vor Ort vorhanden sind, werden die realen Werte wiefolgt gewählt: $R_1=2,7k\Omega\\ R_2=150k\Omega$, da sie den nächsten Größen in der E12 Reihe entsprechen.


In der Folgenden Abbildung sit die Simulation des Filters mit den Realenbauteilwerten zu sehen.
Die Grenzfrequenz des Filters (-3dB) liegt diesmal mit 104Hz etwas über den ursprünglichen 100Hz. da wir die Werte von $R_5$ und $R_6$ nicht verändert haben liegt die Verstärkung bei 10Hz immernoch bei exakt 33,6dB. Bei 3,9 kHz im Diagramm gut zu erkennen wird trotz der höheren eine höhere Dämpfung als vorher erreicht. Diese liegt bei 33,7dB, daran kan mann erkennen das es sich nicht mehr um einen idealen Butterworthfilter handelt. 
%%Verstärkung: 33,6dB gewünscht:
%%20*log(1/48)=-33,6

%%dämpfung bei 3,9Khz = 30,7dB
%% 33,6+30,7 =64,3 gewüncht: 55,9
%% Grenzfreqeunz = 104Hz (-3db)
\begin{figure}[H]
\centering
\begin{gnuplot}[terminal=pdf]
  set nokey 
  set xrange [10:3900]
  set xlabel 'Frequenz in [Hz]'
  set ylabel 'Verstärkung in [dB]'
  set logscale x 10
  plot 'Simulation/Filter_real_frequenzgang.csv' with line
\end{gnuplot}
\caption{Frequenzgang des berechneten Filters mit finalen Bauteilwerten}
\label{plott:filter_freq_real}
\end{figure}


Im der folgenden Abbildung [\ref{plott:filter_sprungantwort}] ist die Antwort des Filters auf ein Rechtecksignal mit 3,9kHz, einem Tastverhältnisvon 0,5 und einer Amplitude von 50mV .Das Überschwingen im Bereich von 7ms ist charakteristisch für den Butterworthfiter und wirks sich negativ auf die Messung des Stromes aus. Allerdings werden solch große Sprünge in der Praxis nicht auftreten werden, da der Strom duch die große Induktivität des Motors nur langsam ansteigt.
\begin{figure}[H]
\centering
\begin{gnuplot}[terminal=pdf]
  set nokey 
  set yrange [0:3]
  set xlabel 'Zeit in [s]'
  set ylabel 'Spannung in [V]'
  plot 'Simulation/Filter_real_time.csv' with line
\end{gnuplot}
\caption{Sprungantwort des Filters}
\label{plott:filter_sprungantwort}
\end{figure}


Die in Abbildung [\ref{plott:ripple}] gut zu Restwellgkeit (Ripple) beträgt 3,36mV,und liegt damit deutlich unter den gerforderten 4,88mV. Als Eingangssignal dient hier ein Rechtecksignal mit 3,9kHz  und einem Tastverhältnisvon 0,5, die Amplitude liegt bei 50mV. Die Tatsache dass das Signal 240mV über den rechnerischen 2,40V  ($0.5V \cdot 48 $) liegt rührt daher dass LT-Spice die steig und fall Zeiten in den low-Bereich des Rechteck signals legt, woduruch der Mittelwert des Signals bei 2,64V liegt.
 
%% Ripple: 0,003358
%% zu hoher Spannungswert durch Steig und Fallzeiten im low bereich des Rechtecksignals
\begin{figure}[H]
\centering
\begin{gnuplot}[terminal=pdf]
  set nokey 
  set xrange [0.03:0.04]
  set yrange [2.62:2.66]
  set xlabel 'Zeit in [s]'
  set ylabel 'Spannung in [V]'
  plot 'Simulation/Filter_real_time.csv' with line
\end{gnuplot}
\caption{Restwellgkeit des Filters}
\label{plott:ripple}
\end{figure}






\chapter{Software}

\section{Übertragungsprotokoll}
Da die Übertragung der Daten via ROS-Serial im ersten Prototypen zu vielen Problemen geführt hat, wurde ien neues Übertragungsprotokoll entwickelt.
Dabei wurde auf Fehlertolleranz niedrigen Ressourcenverbrauch geachtet. Der Datendurchsatz muss hier legendlich ausreichend sein, also alle Daten mit 50Hz
übertragen können.

Der Grundlegende Ablauf der Datenübertragung ist in den Abbildungen [\ref{fig:uC_read}] und [\ref{fig:uC_write}] zu sehen.



\begin{figure}[ht]
\centering
\includegraphics[page=1,width=.8\textwidth]{graph/read.pdf} 
\caption{Lese Daten}
\label{fig:uC_read}
\end{figure}


\begin{figure}[ht]
\centering
\includegraphics[page=1,width=.8\textwidth]{graph/write.pdf} 
\caption{Schreibe Daten}
\label{fig:uC_write}
\end{figure}

\chapter{Evaluierung}



\section{Stormmessung}

Zur Qualitätsbewertung der Stommessung wird das Fahrzeug aufgebockt. 
Das Fahrzeug befand sich während aller Messungen im Akkubetrieb, bei ca. 16,8V Akkuspannung.
Motor wurde, soweit nicht anders beschrieben mit einem PWM-Tastgrad von 30:256 angesteuert.
Zur Beurteilung der Messergebnisse wurden Messungen an diversen Punkten vorgenommen und mit der Ausgabe des Mikrocontrollers verglichen.

Zuerst wurde dazu die Spannung direkt am Shuntwiderstand gemessen.
Dabei kann der periodische Ansteig der Spannung am Shunt beobachtet werden, zu sehen in \cref{fig:filter_eingang}. Die Spannung am Shunt ist Proportional zum Storm durch den Motor.

\begin{figure}[H]
\centering
\includegraphics[width=.8\textwidth]{filter_eingang_mak.png}\\
\caption{Spannung am Shunt}%
\label{fig:filter_eingang}
\end{figure}

Die mittlere Spannung aus \cref{fig:filter_eingang} lässt sich abschätzen, in dem die Spannug in der Mitte eines Impulses abliest und sie mit dem Tastverhältnis multipliziert.
$\SI{7}{\mV}\cdot\frac{30}{256}=\SI{0,82}{\mV} $
Mit Hilfe dieses Mittelwertes können wir grob abschätzen ob der nachfolgende Filter seine Funktion erfolgreich erfüllt.


Als nächstes wird die Spannung nach der Filterschaltung direkt am Operationsversärker gemessen. Es fällt auf das das Signal trotz Teifpass noch Störungen in Form des Motorstromes enthält.
Dies ist eine Auswirkung der gestörten Betriebsspannung, welche später untersucht wird.


\begin{figure}[H]
\centering
\includegraphics[width=.8\textwidth]{filter_ausgang.png}\\
\caption{Spannung nach dem Filter}%
\label{fig:filter_ausgang}
\end{figure}


Wie in \cref{fig:filter_ausgang} zu sehen wurde die Spannung erfolgreich verstärkt und hat etwa eine Spannung von \SI{40}{\mV}. Dividiert durch den Verstärkungsfaktor 48, ergibt sich eine Spannung von 
\SI{0,83}{\mV}. Welche sehr nahe an dem vorher abgeschätzten Wert von \SI{0,82}{\mV} liegt. Der Strom lässt sich nun mit Hilfe des ohmnschen Gesetzes $U=R\cdot I$ errechnen.
\begin{align*}
I=\frac{\SI{0,83}{\mV}}{\SI{0,005}{\ohm}}=\SI{166,7}{\mA}
\end{align*}

Der vom Microcontroller ausgegebenne Strom beträgt im Mittel \SI{156,1}{\mA}.

Zur weiteren Bewertung der Messung wurden jeweils ca 1000 Samples an Daten zu unterschiedlichen PWM-Tastgraden aufgezeichnet.
Dabei soll untersucht werden, ob die Qualität der Messwerte mir zunehmender Geschwindigkeit stabil bleibt.
Die Ergebnisse der Messungen sind in \cref{tab:current_noload} zu sehen.

\begin{table}[H]
  \centering
  \begin{tabularx}{\textwidth}{|X|r|r|}
    \hline
    Tastgrad & Erwartungswert [A] & Standardabweichung [A]  \\ \hline \hline
    10 & 0.0044 & 0.0212\\ \hline
    20 & 0.0730 & 0.0214\\ \hline
    30 & 0.1600 & 0.0215\\ \hline
    40 & 0.2629 & 0.0214\\ \hline
    50 & 0.3653 & 0.0185\\ \hline
    60 & 0.4720 & 0.0231\\ \hline

  \end{tabularx}
  \caption{Motorstorm im Leerlauf}%
  \label{tab:current_noload}
\end{table}

Die Standardabweichung der Messwerte scheint stabil zu bleiben. Da der Strom sich in diesen Messungen in einem sehr niedrigen Bereich befindet, wurden
die Messungen erneut unter Last durchgeführt, der Motor befand sich währenddessen im Kurzschlussbetrieb.

\begin{table}[H]
  \centering
  \begin{tabularx}{\textwidth}{|X|r|r|}
    \hline
    Tastgrad & Erwartungswert [A] & Standardabweichung [A]  \\ \hline \hline
    10 & 0.0454 & 0.0183\\ \hline
    20 & 0.4635 & 0.0243\\ \hline
    30 & 1.1694 & 0.0338\\ \hline
    40 & 1.9822 & 0.0563\\ \hline
    50 & 3.8737 & 0.0477\\ \hline
    60 & 4.7023 & 0.0706\\ \hline
  \end{tabularx}
  \caption{Motorstrom unter Last}%
  \label{tab:current_load}
\end{table}

Der Tabelle \ref{tab:current_load} kann leicht entnommen werden, das die Ströme unter last stark ansteigen. Die Standardabweichung steigt im Verhältnis zum Strom nur leicht.
So dass die Streuung der Messung mit steigendem Strom zwar zunimmt, die relative Abweichung der Werte jedoch nicht steigt.


\section{Spannungsversorgung}

Während der Untersuchung der Strommessung sind Störungen im Ausgabesignal der Filterschaltung festgestellt wurden. Um die Ursache der Störung zu finden wurde die Spannungsversorgung
des Fahrzeuges untersucht.

\begin{figure}[H]
\centering
\includegraphics[width=.8\textwidth]{5V_supply.png}\\
\caption{Störungen im 5V Netz}%
\label{fig:5V Supply}
\end{figure}

Betrachtet man den Verlauf der Spannung im 5V-Netz lässt sich leicht erkennen, dass das Signal von einem dem Strom sehr ähnlichen Signal überlagert wird.
Eine solche Störung kann nur von der Betriebsspannung kommen, da der Motor nicht an das 5V-Netz angeschlossen ist. Scheinbar ist der 5V-Schaltregler nicht in der Lage diese Störungen vollständig auszugleichen.
Daher wird nun die Betriebsspannung untersucht um das Ausmaß der Störungen Beurteilen zu können.
Die Werte wurden in einer 1:10 Teilung gemessen, daher sind alle Spannungen mit dem Faktor 10 zu multiplizieren.

Im Akkubetrieb (\cref{fig:accu_supply}) kann gut Beobachtet werden, dass die Spannung um mehr als \SI{100}{\mV} schwankt. Die Frequenz der Spannungseinbrüche entspricht in etwa
der PWM Frequenz von \SI{7812,5}{\hertz}.  Der Motor wurde währenddessen mit einem Tastgrad von 30:256 Angesteuert und wurde nicht belastet.


\begin{figure}[H]
\centering
\includegraphics[width=.8\textwidth]{VCC_AKKU.png}\\
\caption{Störungen der Betriebsspannung mit Akku}%
\label{fig:accu_supply}
\end{figure}


Vergleicht man die Messungen im Akkubetrieb mit den Messungen im Netztbetrieb (\cref{fig:power_supply}), kann man erkennen das das Netzteil versucht die Schwankungen auszugleichen.

\begin{figure}[H]
\centering
\includegraphics[width=.8\textwidth]{VCC_SUPPLY.png}\\
\caption{Störungen der Betriebsspannung mit Netzteil}%
\label{fig:power_supply}
\end{figure}

Unter Last bei einem Tastgrad von 50:256 betägt die Amplitude der Schwankungen im Akkubetrieb über 3V. Die Messung wurde ohne 1:10 Teilung durchgeführt.
Erstaunlicherweise war trotz der starken Schwankungen noch ein stabiler Betrieb des Fahrzeuges möglich. Dies wurde jedoch nur ca 30sec getestet, denn
die hohe Belastung führte zu einer starken Wärmeentwicklung des Motors.

\begin{figure}[H]
\centering
\includegraphics[width=.8\textwidth]{VCC_AKKU_LAST.png}\\
\caption{Störungen der Betriebsspannung mit Akku unter Last}%
\label{fig:power_supply}
\end{figure}

Betrachet man das 5V-Netz unter diesen Bedingungen (\cref{fig:5V_last}) zeigt sich die beeindruckende Leistung des 5V Schaltreglers.
Das 5V-Netz scheint troz der immensen Störung stabiler zu sein als bei niedriger Motorlast. 

\begin{figure}[H]
\centering
\includegraphics[width=.8\textwidth]{5V_LAST.png}\\
\caption{Störungen des 5V Netzes unter Last}%
\label{fig:5V_last}
\end{figure}

Da das 5V Netz auch unter Last stabil bleibt, sind auch bei höheren Motorlasten keine größeren Störungen in den Sensormesswerten zu erwarten.


\section{Inertisalsensor}

Da die Auswertung der Inertialsensorik ist nicht teil diese Arbeit. Um jedoch die Abhängigkeit der Inertialsensorik von der Motorlast
zu untersuchen, wurden einige Messwerte aufgenommen.

Die Messwerte wurden unter Last bei dem angegebenen Tastgrad aufgenommen. Um die Anzahl der Messwerte gering zu halten wurde legendlich die x-Achse der Sensoren untersucht.

\begin{table}[H]
  \centering
  \begin{tabularx}{\textwidth}{|X|r|r|}
    \hline
     Tastgrad & Erwartungswert [\SI{}{\metre\per\second\squared}] & Standardabweichung [\SI{}{\metre\per\second\squared}]  \\ \hline \hline
     20 & -0.02034  & 0.00078 \\ \hline
     30 & -0.02047  & 0.00225 \\ \hline
     40 & -0.02035  & 0.00097 \\ \hline
     50 & -0.02034  & 0.00074 \\ \hline
     60 & -0.02028  & 0.00189 \\ \hline
  \end{tabularx}
  \caption{Accelerometer}%
  \label{tab:acc}
\end{table}

\begin{table}[H]
  \centering
  \begin{tabularx}{\textwidth}{|X|r|r|}
    \hline
     Tastgrad & Erwartungswert [\SI{}{\radian\per\second}] & Standardabweichung [\SI{}{\radian\per\second}]  \\ \hline \hline
     20 & 0.72961 & 0.27632\\ \hline
     30 & 0.77174 & 0.30825\\ \hline
     40 & 0.78556 & 0.27121\\ \hline
     50 & 0.77792 & 0.27066\\ \hline
     60 & 0.87919 & 0.23369\\ \hline
  \end{tabularx}
  \caption{Gyroskop}%
  \label{tab:gyro}
\end{table}


Sowohl beim Accelerometer als auch beim Gyroskop lässt sich anhand der Daten keine Abhängigkeit erkennen.

\begin{table}[H]
  \centering
  \begin{tabularx}{\textwidth}{|X|r|r|}
    \hline
     Tastgrad & Erwartungswert [\SI{}{\tesla}] & Standardabweichung [\SI{}{\tesla}]  \\ \hline \hline
     20 & 0.06974 & 0.00436\\ \hline
     30 & 0.07602 & 0.00715\\ \hline
     40 & 0.07336 & 0.01154\\ \hline
     50 & 0.06504 & 0.01366\\ \hline
     60 & 0.06564 & 0.01894\\ \hline
  \end{tabularx}
  \caption{Magnetometer}%
  \label{tab:mag}
\end{table}

Beim Magnetometer lässt sich jedoch erkennen, das die Streung der Daten mit zunehmender Motorlast ebenfals zunimmt. Da der Motor jedoch ein Magnetfeld
erzeugt, welches mit zunehmender Belastung zunimmt, war ein Einfluss auf die Messwerte zu erwarten. Diese Störungen sollten bei der verwendung
des Magnetometers beachtet werden.


  %set xrange [0.03:0.04]
  %set yrange [2.62:2.66]
  %set xlabel 'Zeit in [s]'
  %set ylabel 'Spannung in [V]'

\begin{figure}[H]
\centering
\begin{gnuplot}[terminal=pdf]
  set nokey 

  plot 'MessData/Imu-servo/mit_stoerung.csv' with line
\end{gnuplot}
\caption{Restwelligkeit des Filters}
\label{plott:ripple}
\end{figure}

\begin{figure}[H]
\centering
\begin{gnuplot}[terminal=pdf]
  set nokey 

  plot 'MessData/Imu-servo/ohne_stoerung.csv' with line
\end{gnuplot}
\caption{Restwelligkeit des Filters}
\label{plott:ripple}
\end{figure}


\section{Stromverbrauch}

Der Stromverbrauch des Fahrzeugs ist ein wichtiges Kriterium in den statischen Disziplinen. Um den Stromverbrauch im laufenden Betrieb messen zu können, wird hier ein Versuchsaufbau verwendet welcher der Messung des
Motorstomes ähnelt. Die Schaltung besteht dabei aus einem Shuntwiderstand, einer aktiven Filterschaltung und einem Arduino, welcher die Daten zum NUC weiterleitet. Der Vorteil dieser Methode ist, dass die Daten
unter realen Bedingungen in Echtzeit aufgezeichtet werden können. \cref{fig:Strom} zeigt den Verlauf des Stromes währed folgendem Szenario: Bis ca 25s steht das Auto, sämmtliche Software ist dabei auf 
dem Fahrzeug aktiv. Ab 25s beschleunigt das Fahrzeug auf $1,3\frac{m}{s}$ und verhart dort bis ca. Sekunde 57, in welcher es gegen eine Wand fährt. Der grüne Graph stellt dabei den Strom durch den Motor dar, während
der blaue Graph den Gesammtverbrauch des Fahrzeugs darstellt. Die roten Linien Stellen einen gleitenden Mittelwert aus den letzten 200 Messwerten dar. Gut zu erkenn ist, das der Stromverbrauch des Fahrzeugs im Stand unter 
10 Watt liegt. Der Mittelwert des Verbrauches im Stand beträgt 7,9 Watt, während das Fahrzeug in der dar fahrt knapp 13 Watt an Leistung aufnimmt. Nur während das Fahrzeug beschleunigt benötigt es für die Dauer
des Beschleunigungsvorganges mehr Leistung. Fährt das Fahrzeug gegen ein Hinderniss, sodas die Räder blockieren befindet sich der Motor im Kurzschlussbetrieb, dabei reduziert sich sein widerstand auf den Ohmschen widerstand
Motors, was zu einem hohen Stromfluss durch den Motor führt. Dauerhaft kann das durch Überhitzung zur Zerstörung des Motors oder der Treiberplatine führen.


\begin{figure}[H]
\centering
\includegraphics[width=.8\textwidth]{Strom/Power.png}\\
\caption{Salle-Key Tiefpass mit Shunt}%
\label{fig:Strom}
\end{figure}






\section{Infrarotsensoren}

\section{Inertialsensor}

\section{Zeitverhalten der seriellen Verbindung}

Um eine Aussage über das Alter eines Messwertes zu machen. 

\cite{ds-at90can}
adc Wandlung=13Takte
Prescaler=128 bei 16MHz=125000kHz =104uS pro Messung


Bytedauer = 11Bit (1Start+8Daten+2Stopp) 500000Baud --> 22us Bytdauer -> Preamble=5Bytes (4x255+ID) = 110us



\subsection{VoltageCurrent}
25.0319430503
718.591761425


\begin{gnuplot}[terminal=pdf]

  n=70 #number of intervals
  max=872. #max value
  min=596. #min value
  width=(max-min)/n #interval width

  hist(x,width)=width*floor(x/width)+width/2.0

  set xrange [min:max]
  set yrange [0:]

  set offset graph 0.05,0.05,0.05,0.0
  set xtics min,(max-min)/5,max
  set boxwidth width*0.9
  set style fill solid 0.5 #fillstyle
  set tics out nomirror
  set xlabel "x"
  set ylabel "Frequency"
  #count and plot
  plot "MessData/VoltageCurrent.csv" u (hist($1,width)):(1.0) smooth freq w boxes lc rgb"grey" notitle
\end{gnuplot}

\subsection{Distance}
28.1616619788
484.734158193


\begin{gnuplot}[terminal=pdf]

  n=70 #number of intervals
  max=673. #max value
  min=359. #min value
  width=(max-min)/n #interval width

  hist(x,width)=width*floor(x/width)+width/2.0

  set xrange [min:max]
  set yrange [0:]

  set offset graph 0.05,0.05,0.05,0.0
  set xtics min,(max-min)/5,max
  set boxwidth width*0.9
  set style fill solid 0.5 #fillstyle
  set tics out nomirror
  set xlabel "x"
  set ylabel "Frequency"
  #count and plot
  plot "MessData/Distance.csv" u (hist($1,width)):(1.0) smooth freq w boxes lc rgb"green" notitle
\end{gnuplot}


\subsection{Infrared}
35.0709303606
696.91743058



\begin{gnuplot}[terminal=pdf]

  n=70 #number of intervals
  max=825. #max value
  min=586. #min value
  width=(max-min)/n #interval width

  hist(x,width)=width*floor(x/width)+width/2.0

  set xrange [min:max]
  set yrange [0:]

  set offset graph 0.05,0.05,0.05,0.0
  set xtics min,(max-min)/5,max
  set boxwidth width*0.9
  set style fill solid 0.5 #fillstyle
  set tics out nomirror
  set xlabel "x"
  set ylabel "Frequency"
  #count and plot
  plot "MessData/Infrared.csv" u (hist($1,width)):(1.0) smooth freq w boxes lc rgb"green" notitle
\end{gnuplot}


\subsection{uCTime}
31.2060346589
477.761007865



\begin{gnuplot}[terminal=pdf]

  n=70 #number of intervals
  max=685. #max value
  min=366. #min value
  width=(max-min)/n #interval width

  hist(x,width)=width*floor(x/width)+width/2.0

  set xrange [min:max]
  set yrange [0:]

  set offset graph 0.05,0.05,0.05,0.0
  set xtics min,(max-min)/5,max
  set boxwidth width*0.9
  set style fill solid 0.5 #fillstyle
  set tics out nomirror
  set xlabel "x"
  set ylabel "Frequency"
  #count and plot
  plot "MessData/ucTime.csv" u (hist($1,width)):(1.0) smooth freq w boxes lc rgb"green" notitle
\end{gnuplot}

\subsection{Motor}
38.0093527069
391.111706657


\begin{gnuplot}[terminal=pdf]

  n=70 #number of intervals
  max=534. #max value
  min=287. #min value
  width=(max-min)/n #interval width

  hist(x,width)=width*floor(x/width)+width/2.0

  set xrange [min:max]
  set yrange [0:]

  set offset graph 0.05,0.05,0.05,0.0
  set xtics min,(max-min)/5,max
  set boxwidth width*0.9
  set style fill solid 0.5 #fillstyle
  set tics out nomirror
  set xlabel "x"
  set ylabel "Frequency"
  #count and plot
  plot "MessData/motor.csv" u (hist($1,width)):(1.0) smooth freq w boxes lc rgb"green" notitle
\end{gnuplot}




\subsection{IMU}
40.8272894054
3425.96398337


\begin{gnuplot}[terminal=pdf]

  n=70 #number of intervals
  max=3605. #max value
  min=3312. #min value
  width=(max-min)/n #interval width

  hist(x,width)=width*floor(x/width)+width/2.0

  set xrange [min:max]
  set yrange [0:]

  set offset graph 0.05,0.05,0.05,0.0
  set xtics min,(max-min)/5,max
  set boxwidth width*0.9
  set style fill solid 0.5 #fillstyle
  set tics out nomirror
  set xlabel "x"
  set ylabel "Frequency"
  #count and plot
  plot "MessData/imu.csv" u (hist($1,width)):(1.0) smooth freq w boxes lc rgb"green" notitle
\end{gnuplot}


\subsection{Gesammt}
142.365993405
6195.08004809


\begin{gnuplot}[terminal=pdf]

  n=70 #number of intervals
  max=6522. #max value
  min=5574. #min value
  width=(max-min)/n #interval width

  hist(x,width)=width*floor(x/width)+width/2.0

  set xrange [min:max]
  set yrange [0:]

  set offset graph 0.05,0.05,0.05,0.0
  set xtics min,(max-min)/5,max
  set boxwidth width*0.9
  set style fill solid 0.5 #fillstyle
  set tics out nomirror
  set xlabel "x"
  set ylabel "Frequency"
  #count and plot
  plot "MessData/gesammt.csv" u (hist($1,width)):(1.0) smooth freq w boxes lc rgb"green" notitle
\end{gnuplot}

\chapter{Ausblick und Fazit}

Im Rahmen dieser Arbeit wurde eine Treiberplatine für das OttoCar geschaffen, ausgehend von einem Anfangsbestand von Komponenten sollte die Platine, diversen Anforderungen genügen.
Die Anforderungen umfassten dabei die Ansteuerung des Motors und des Servomotors, welche erfolgreich umgesetzt wurden, und zuverlässig ihren Dienst verrichten.
Auch die Sharp GP2D Sensoren konnten erfolgreich in das System integriet werden....................................................Evaluierung Sensorqualität
Die benötigte Beleuchtung wurde duch einen LED-Streifen mit WS2812 Contollern umgesetzt, welche durch ihre Bauart eine einfache integration in das Auto ermöglichen.
Da diese als RGB-Leds ausgeführt sind Übertreffen sie die Anforderungen sogar, da sie nahezu jede mögliche Farbe Annehmen können. Innerhalb der Software können diese einzeln Angesteuert
werden und so zu Fehlerdiagnose genutzt werden. Bei der Integration des Sparkfun Inertialsensors traten Probleme mit dem Magnetometer auf, welche im nachhinein nur provisorisch abgemildert werden konnten.
Trotz diese Probleme kann der Intertialsensor erfolgreich ???????????????? zur Inertialnavigation eingesetzt werden %%%%%%%%\cite{Martins_arbeit}}

Die Auslegung der 5V Stomversorgung kann ebenfalls als erfolg gewertet werden, der Regler schafft es trotz ernormer Schwankungen der Betriebsspannung die Spannung, bis auf minimale Störungen
stabil zu halten, so dass es zu keiner Störung der Sensormesswerte kommt. Die Odometrie konnte die Anforderungen leider nicht vollständig erfüllen, da nur eine Wegmessug nach vorne möglich ist.
Die Qualität der Geschwindigkeitsangabe ist dafür auch bei Mittleren Geschwindigkeiten bis \SI{2}{\metre\per\second} auch in Kurven gut brauchbar. Durch die Fusion, der
Inertailsensorik und Odometrie konnte ein Inertialnavigationssystem
%%%%%%%%\cite{Martins_arbeit}}
entwickelt werden, welche eine gute????????????????? Aussage über die Postion des Autos machen kann.

Als großer Erfolg, kann die Energieeffizeins des Systems gewertet werden! 
In zukü+nftigen entwicklingenn sollte die Leistungselkektronik jedoch von ser Sensorkmgetrennt werden, um Störungen zu vermeiden

% \item nötige Elektonik zur Motoransteuerung (vor und Rückwärts)
% \item Anschluss und Steuerung für einen Servomotor
% \item Anschlusse und Steuerung für die nötige Beleuchtung
% \item Anschlüsse für die Sharp GP2D Sensorn
% \item Integration des Sparkfun Inertialsensors
% \item Leistungsfähige Stromversorgung

\include{Ausblick}

\chapter*{Eidesstattliche Erklärung}
Ich erkläre, dass ich meine Bachelor-Arbeit sdsds selbstständig und ohne Benutzung anderer als der angegebenen Hilfsmittel angefertigt habe und dass ich alle Stellen, die ich wörtlich oder sinngemäß aus Veröffentlichungen entnommen habe, als solche kenntlich gemacht habe. Die Arbeit hat bisher in gleicher oder ähnlicher Form oder auszugsweise noch keiner Prüfungsbehörde vorgelegen.\\\\

Ich versichere, dass die eingereichte schriftliche Fassung der auf dem beigefügten Medium gespeicherten Fassung entspricht.


\noindent Magdeburg, den \today
\begin{flushright}
$\overline{~~~~~~~~~\mbox{Julian-B. Scholle}~~~~~~~~~}$
\end{flushright}


\listoffigures

\bibliography{BIB}{}
\bibliographystyle{alpha}

\end{document}