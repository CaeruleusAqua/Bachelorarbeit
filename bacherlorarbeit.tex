\documentclass[11pt,twoside,a4paper,openright]{mpreport}
\usepackage[utf8]{inputenc}
\usepackage[ngerman]{babel}
%\usepackage{a4wide}
%\usepackage[T1]{fontenc}
%\usepackage{textcomp}
%\usepackage{mathptmx}
\usepackage[scaled=0.85]{helvet}
%\usepackage{courier}           % if you really want to use Courier
%\usepackage{url}
\usepackage{cite}
\usepackage{tabularx}
\usepackage{amsmath}
\usepackage{amssymb}
\usepackage[ 
   colorlinks,        % Links ohne Umrandungen in zu wählender Farbe 
   linkcolor=black,   % Farbe interner Verweise 
   filecolor=black,   % Farbe externer Verweise 
   citecolor=black    % Farbe von Zitaten 
]{hyperref}
\usepackage{ngerman}
\usepackage{textcomp}
\usepackage{wdok-title}
\usepackage{tikz,pgfplots}
\usepackage[section]{placeins} 
\usepackage{here} 
\usetikzlibrary{patterns}
\usetikzlibrary{fpu}
%%\usepackage[pdf]{pstricks}
\setlength{\parindent}{0pt}
\newcommand{\arcosh}{\mbox{arcosh }}
%\pgfplotsset{compat=1.8}
\definecolor{lila}{rgb}{0,0.2,0.8}

% Examples for the definition of convenience commands
\newcommand{\package}[1]{\texttt{#1}}
\newcommand{\foreign}[1]{\emph{#1}}
\newcommand{\q}[1]{»#1«}

% Scale Courier by 0.9
% cf. <http://groups.google.de/groups?selm=yfid76obspu.fsf@triumf.ca>
%\DeclareFontFamily{T1}{pcr}{}
%\DeclareFontShape{T1}{pcr}{m}{n}{
%   <-> s*[.9]pcrr8t
%}{}

\title{Hardware/Software Codesign}
\author{Julian-Benedikt Scholle}
\supervisor{Dr. Ing. Sebastian Zug\\
  Dipl.-Inform Christoph Steup}

\begin{document}
\maketitle
\tableofcontents

\chapter{Einleitung}

\chapter{Anforderungsanalyse} Anforderungsanalyse
\subsection{Anforderungen an die Platine}
Laut Regelwerk sind alle Teams zur verwendung eines elektrischen Antriebs verpflichtet.
Die Anzahl der angetriebenen Räder ist nicht vorgeschrieben
des weiteren muss das Auto durch Akkus mit Strom versorgt werden.
Die Übertragung von Daten ist während der Dauer der Disziplinen nicht gestattet

\subsubsection{Fahrzeugabmessungen}
Es ist ein vierrädriges Fahrzeug mit 2 Achsen im Maststab 1:10 vorzusehen. Die Spurweite, gemessen von Reifenmitte zu
Reifenmitte, muss mindestens 16 cm betragen. Der Radstand muss mindestens 20 cm betragen.
Die Höhe des Fahrzeuges darf 30 cm nicht überschreiten, über das Fahrzeug hinausragende flexible Antennen sind gestattet.
Zur Abnahme des Fahrzeuges, muss es durch ein 40 cm Breites und 30 cm Hohes Tor fahren.


\subsubsection{RC-Modus}
In Notsituationen muss es möglich sein das Fahrzeug mit Hilfe einer Funkfernbedienung anzuhalten und manuell zu steuern. Eine solche Notsituation tritt ein, wenn
das Auto seine Aufgabe aufgrund eines Fahrfehlers oder anderem Fehlverhalten nicht mehr autonom fortführen kann.
Der RC-Modus muss per Fernbedienung eingeschaltet und ausgeschaltet werden, bei Aktivierung des RC-Modus muss das Fahrzeug unverzüglich angehalten werden.
Während des Wettbewerbs darf die Geschwindigkeit des Autos $0,3\frac{m}{s}$ nicht überschreiten.
Da das 2,4-GHz Band bereits durch die Vorort genutzte Kameratechnik belegt ist können diese Frequenzen nicht für den RC-Modus genutzt werden.
Der RC-Modus muss durch eine blaue Leuchte an der höchsten stelle des Fahrzeuges angezeigt werden, welche mit einer Frequenz von 1-Hz blinkt.

\subsubsection{Signalleuchten}
Durch die Anlehnung des Wettbewerbes an den realen Straßenverkehr muss das Auto über alle in echten Auto vorhandene Signalleuchten besitzen. 
Dazu gehören 3 rote Bremslichter am Heck des Autos sowie jeweils 2 gelbe Blinker Rechts und Links am Fahrzeug.  Die Blinkfrequenz der Blinker muss
1-Hz betragen.

\subsubsection{Verkleidung}
Um eine schnelle Wartung und Überprüfung zu ermöglichen muss die Fahrzeug Abdeckung jederzeit schnell und einfach entfernt werden können. Des Weiterem muss die dem 
Schutzgrad IP 11 entsprechen.


\subsection{Resultat aus den Anforderungen}
Aufgrund der Anforderungen der elektrischen Antriebes und des Maststabes von 1:10 kann als Grundaufbau des Autos ein gängiges Ferngesteuertes Auto aus dem
freien Handel genutzt werden. Die zur Verfügung stehenden Möglichkeiten wurden dabei im Team diskutiert und die Wahl fiel dabei auf das ``TT-01R Type E'' welches
allen Anforderungen genügt. Tamiya ist seit langen ein im Modellbau etablierter Hersteller so das hier Ersatzteile lange verfügbar sein sollten.

Im Bausatz des TT-01R Type E ist bereits ein Elektromotor samt Reglerelektronik enthalten. Leider verbaut Tamiya Motoren von vielen verschiedenen Herstellern,
zu denen es leider kein Datenblatt gibt. Einige Recherchen ergaben allerdings das alle Motoren Nachbauten eines ``Mabuchi RS-540SH'' sind, von welchem ein Datenblatt
existiert \cite{Mabuchi}








\chapter{Stand der Technik}



\include{Auto}

\include{Motortreiber}


\chapter{Motorstrommessung am Shunt}


\section{Problem}

An einem mit PWM angesteuertem DC-Motor soll eine Strommessung mit Hilfe eines Shuntwiederstandes
durchgeführt werden. Aufgrund der PWM Ansteuerung muss der DC-Anteil aus dem Signal herausgefiltert werden!


\section{Prinzip der Strommessung}

Die Messspannung wird über einen Shuntwiederstand zur Masse gemessen! Aufgrund nicht vorhandener Datenblätter des Motors
wird von einem expirimentel Ermittelten maximalen Strom des Motors ausgegangen. Dieser beträgt bei einer Betriebsspannung von 20V ca. 20A.
Da einen Shunt mit einer maximalen Belastbarkeit von 2 Watt eingesetzt wird, darf der maximale Spannungsabfall am Shunt 100mV nicht überschreiten.
Nach dem Ohmschen Gesetz ergibt sich dadurch ein Widerstand von $0,005 \Omega$  für den Shunt. Shuntwiederstände in der Größe sind problemlos zu bekommen.
Da es sich hier um eine Worst Case Rechnung handelt, wird der zusätzliche Widerstand des Shuntwiederstandes und der damit verringerte Strom bewusst ignoriert.

Die über den Shuntwiederstand gemessene Spannung soll über den ADC Eingang des Mikrocontrollers eingelesen werden. Vorher jedoch muss das Signal gefiltert werden, da der Strom
durch die Ansteuerung mittels der Pulsweitenmodulation nicht konstant ist!



\section{Anforderungen}
Die maximale Auflösung des Mikrocontrollers soll ausgenutzt werden. Der ADC des Mikrocontrollers arbeitet mit einer Auflösung von 10 Bit und einer 
Referenzspannung von 5V. Um die Auflösung des ADC auszunutzen muss das Signal, aufgrund unseres Spannungsabfalls verstärkt werden.

Als Anforderung ergibt sich außerdem, dass der maximale Ripple des Endsignals kleiner ist als der Quantisierungsfehler des ADC.
So ist es möglich aufeine zusätzliche digitale Filterung weitgehend zu verzichten.
Die kleinst mögliche zu erfassende Spannung des ADC beträgt $\frac{5}{2^{10}}=4,88mV$.
Diesen wert sollte der Ripple des Endsignales nicht überschreiten.
Aus einem möglichst kleinem Ripple resultiert eine möglichst hohe Filterordnung bzw. eine niedrige Grenzfrequenz.
Allerdings soll $U_{DC}$ einer Änderung des Mittelwertes, also einer Änderung des Tastverhältnisses, möglichst
schnell folgen. Diese Anforderung widerspricht der Vorherigen, so das ein Kompromiss gefunden werden muss.

\section{Bestimmung des Filtertyps}
Aufgrund des sehr niedrigen Spannungspegels am Shunt,ist es nötig das Signal zu verstärken. Da zum verstärken des Signals aktive elektronische Elemente notwendig sind,
z.B. ein Operationsverstärker, wird an dieser Stelle gleich ein aktiver Filter verwendet. Dieser gibt uns die Möglichkeit des Messignal zu verstärken und gleichzeitig zu
Filtern. Da wir als unser Signal im optimalen Fall eine Gleichspannung darstellt müssen wir die Hochfrequenten Anteile unseres Signales herausfiltern, dies geschieht 
mit Hilfe eines Tiefpasses. Es gibt im Grunde 2 übliche aktive Tiefpässe, den Tiefpass mit Mehrfachgegenkopplung und den Sallen-Key Filter. Ersterer verwendet
einen invertierenden Verstärker, dieser invertiert das Messsignal. Da der \textmu Controller jedoch nur mit positiven Spannungen umgehen kann müsste man hier mit einer 
negativen Referenzspannung Arbeiten, was den Schaltungsaufwand unnötig vergrößern würde. Der Sallen-Key Tiefpass benutzt einen nicht invertierenden Verstärker, welcher diesen
Nachteil nicht hat. So dass ab dieser Stelle ein Sallen-Key Tiefpass entwurfen wird.

%%TODO Quellen (verweise auf übliche Filtertypem)

\section{Die Filterschaltung}

wie im vorherigen Abschnitt diskutiert wird hier ein Sallen-Key Tiefpass entwurfen. Zum Entwurf der Schaltung wurde Eagle genutzt.

\begin{figure}[H]
\centering
\includegraphics[width=.8\textwidth]{filter_schaltung.png}\\
\caption{Salle-Key Tiefpass mit Shunt}%
\label{fig:fschalt}
\end{figure}



\section{Dimensionierung des Verstärkers}

In bisherigen Rechnungen wurde ein maximaler Spannungsabfall von 100mV am Shunt errechnet. Da der Messbereich des voll ADC ausgenutzt werden soll,
ist es nötig das Messsignal zu verstärken. Hierzu wir ein Nichtinvertierender Verstärker benutzt. Da der Messbereich des ADC bis 5V reicht, wird hier eine 
50 fache Verstärkung angestrebt.

Für einem Nichtinvertierenden Verstärker ergibt sich dann:

\begin{align*}
v &= 1 + \frac{R_{F3}}{R_{F4}}\\
50 &= 1 + \frac{R_{F3}}{R_{F4}}\\
49\cdot R_{F4} &= R_{F3}
\end{align*}
\\
Wobei $R_{F4} = 47 k\Omega$ und $R_{F3} = 1 k\Omega$  gewählt werden, was eine Verstärkung von 48 ergibt.

%%TODO Warum werde widerstände so gewählt

\section{Anforderungen an den Filter}

\begin{figure}[H]
\centering
\includegraphics[width=.8\textwidth]{oszi.png}\\
\caption{Spannung am Shunt + PWM}%
\label{fig:pwm+i}
\end{figure}

Da dem Messsignal wie in Abbildung \ref{fig:pwm+i} zu erkennen, die PWM Frequenz zu Grunde liegt wird sich bei der Dimensionierung des Filters einer Idee nach \cite{Alter2008} bedient, nach der die maximale Amplitude des Ripple der Grundschwingung bei einem
Tastverhältnis von 0,5 entspricht. Die Amplitude der Grundschwingung ergibt sich aus dem ersten Koeffizienten der Fourierreihe einer Rechteckschwingung.
\begin{align}
A_1 = K\cdot \frac{1}{\pi}[\sin(\pi p)-\sin(2\pi(1-\frac{p}{2}))]
\label{eq:ripple}
\end{align}
Wobei $p$ dem Tastverhältnis und $K$ der maximale Amplitude des Ursprungsingals entspricht \cite{Alter2008}. $K$ entspricht den errechneten 100mV multipliziert mit dem Verstärkungsfaktor 48, also 4,8V. Das Tastverhältnis $p$ wird zu 0,5
angenommen. Mit (\ref{eq:ripple}) ergibt sich für die Amplitude der Grundschwingung $ A_1 = K\cdot \frac{2}{\pi} = 3,056V$. $A_1$ soll auf $ < 4,88mV$ gedämpft werden.
Als Sperrfrequenz $\Omega_s $ wird hier die PWM Frequenz angesetzt. Für $H(\omega=2\pi f_{PWM})$ gilt also:

\begin{align}
H(\omega=2\pi f_{PWM}) \le \frac{4,88mV}{3,056V} \mathop{\hat{=}} 20\cdot\log(\frac{4,88mV}{3,056V})= -55,9 dB
\label{eq:daempfung}
\end{align}

Da das Projekt möglichst kostensparend durchgeführt werden soll, also auch Bauteilsparend, wird im Folgenden von den üblichen Konventionen zur dimensionierung von Filtern abgewichen.
Statt eine fixe Grenzfreqeunz festzulegen und die benötigte Filterordnung zu bestimmen, wird die Filterordnung vorgegeben und die Grenzfrequenz variiert.

\section{Filterentwurf}

\subsection{Bestimmung des Filtertyps}

Des Filtertyp muss in zweierlei Hinblick bestimmt werden. Einmal im hinblick auf die Schaltung und seinem Frequenzgang.
Im groben gibt es 2 mögliche aktive Tiefpassfilterschaltungen, den Sallen-Key Teifpass mit nicht invertierendem OPV und dem aktiven Tiefpass mit Mehrfachgegenkopplung 
(invertierender OPV). Der aktive Tiefpass mit Mehrfachgegenkopplung benötigt allerdings negative Spannungsniveaus die auf der Treiberplatine nicht zur
verfügung stehen, deshalb wird an dieser Stelle nur der Sallen-Key Teifpass betrachtet.
Was den Frequenzgang angeht gibt es viele Filtercharakteristiken, eine Auswahl an haufig verwendeten Charakteristiken wird hier verglichen.

Der \emph{Butterworth}-Filter besitzt einen maximal flachen Verlauf des Frequenzganges im Durchlassbereich und eine monoton verlaufende Dämpfung im Sperrbereich.
Leider hat der Butterworth-Filter nur eine geringe Flankensteilheit im Sperrbereich (20dB/Dekade pro Ordnung). Ein Butterworth-Filter 1. Ordnung entspricht einen  einfachen RC-Filter.

Der \emph{Tschebyscheff}-Filter hat eine höhere Flankensteilheit als der Butterworth-Filter, allerdings entsteht beim Tschebyscheff-Filter Welligkeit im Durchlassbereich,
welche mit höherer Ordnung zunimmt. Durch die Welligkeit im Duchlassbereich würde ein zusätzlicher Ripple im Signal entstehen, weshalb der Tschebyscheff-Filter nicht
für den geforderten Filter geeignet ist 

Der \emph{Bessel-Filter} hat den Vorteil einer konstanten Gruppenlaufzeit, hat dafür aber eine noch geringere Flankensteilheit als der Butterworth-Filter.
Da eine konstante Gruppenlaufzeit für den geforderten Filter nicht von Vorteil ist, da das Endsignals einer Gleichspannung entsprechen sollte, ist der Butterworth-Filter
die bessere Wahl.


\begin{figure}[H]
\centering
\begin{tikzpicture}
	\draw[->,thick] (0,0) -- (7.5,0) node[right] {$f[\text{Hz}]$};
	\draw[->,thick] (0,0) -- (0,3.3) node[above] {$a[\text{dB}]$};
	\draw (0,2.5)node[left] {$a_{\text{min}}$} (-0.1,2.5)--(2.9,2.5);
	\draw (0,1)node[left] {$a_{\text{max}}$};
	\def \bsp{(0,1)--(1,1)--(1,2.4)--(1,2.4)--(0,2.4)}
	\draw (-0.1,1)--(1,1)--(1,2.4) (1,0)node[below] {$f_g$};
	\pattern[pattern=north east lines] \bsp;
	\draw[dashed] (1,1) -- (1,-0.1);
	\def \bsd{(3,0) -- (3,2.5) -- (7,2.5) -- (7,0)}
	\pattern[pattern=north east lines] \bsd;
	\draw (3,0)node[below] {$f_s$} -- (3,2.5) -- (7,2.5);

\end{tikzpicture}
\caption{Tiefpass Toleranzfeld}%
\label{fig:analog}
\end{figure}
Für unsere Schaltung wird ein Sallen Key Tiefpass 2. Ordnung entwurfen. Die PWM-Frequenz $f_{PWM}$ beträt 3,9kHz.
Die Sperrfrequenz entspreicht der PWM Frequenz, also der Frequenz unserer Grundschwingung. $\Omega$ entspricht der mit der Grenzfreqeunz 
normierten Frequenz $\Omega=\frac{f}{f_g}$. Nach (\ref{eq:daempfung}) ergibt sich für Abbildung \ref{fig:analog}
$f_s=f_{PWM}=3,9 kHz$, $a_{min}=55,9 dB$ und $a_{max}$ wird auf 3dB festgelegt.



\subsection{Butterworth}
\subsection{Bestimmung der Grenzfreqeunz}
\begin{align}
n \ge \frac{\log{\sqrt{\frac{e^{2a_{min}}-1}{e^{2a_{max}}-1}}}}{\log{\Omega_s}}
\label{eq:butterworth}
\end{align}
Die Filterordnung nach Butterworth wird nach (\ref{eq:butterworth}) bestimmt. Umgestellt nach $\Omega_s$ ergibt sich:

\begin{align}
\Omega_s \le  \left(\frac{e^{2a_{min}}-1}{e^{2a_{max}}-1}\right)^{\frac{1}{2n}}
\end{align}



Für die Berechnung der Sperrfrequenz $\Omega_s$ müssen  $a_{min}$ und $a_{max}$ in Neper umgrechnet werden. Wobei:
\begin{align*}
1 \text{dB} =  \frac{\ln{10}}{20}\text{Np} = 0,115129255 \text{Np}   
\end{align*}

Damit ergibt sich für $a_{min}=55,9 dB\cdot \frac{\ln{10}}{20}=6,45Np$ und für  $a_{max}=3 dB\cdot \frac{\ln{10}}{20}=0,345Np$. Die Filterordnung wird auf 2 festgelegt.
\begin{align}
\Omega_s \le  \left(\frac{e^{2\cdot6,45N }-1}{e^{2\cdot 0,345Np}-1}\right)^{\frac{1}{2n}}  = 35,8
\end{align}

Die Grenzfreqeunz $f_g$ ergibt sich jetzt aus:

\begin{align}
\frac{f_s}{\Omega_s} \le \frac{3,9kHz}{35,8} = 108,9Hz
\end{align}

\subsubsection{Filterentwurf}
Im voherigen Abschnitt wurde berechnet das die Grenzfreqeunz der Filters kleiner als 108,9Hz sein muss.
Im Folgenden wird nun ein Sallen-Key Filter 2. Ordnung mit einer Grenzfrequenz von 100Hz entwurfen.
Die genaue Wahl der Grenzfreqeunz ist hier nicht relevant da die realen Bauteile nicht in  allen Größen 
verfügbar sind und daher am Schluss variiert werden müssen, wodurch sich die Grenzfrequen des Filters leicht ändert.


\subsubsection{Finaler Entwurf}



Betrachten wir das Polstellen-Nullstellendiagramm eines Butterworth Filters 2. Ordnung, wie in Abbildung [\ref{fig:filter_polnul}]


\begin{figure}[H]
\centering
\begin{tikzpicture}
	\draw[->,thick] (-3,0) -- (3,0) node[right] {$\text{Re}$};
	\draw[->,thick] (0,-3) -- (0,3) node[above] {$\text{Im}$};
	\draw[dashed,red,very thin] (-3,2) -- (3,2);
	\draw[dashed,red,very thin] (-3,-2) -- (3,-2);
	\draw[dashed,red,very thin] (2,-3) -- (2,3);
	\draw[dashed,red,very thin] (-2,-3) -- (-2,3);
	\draw[dashed,blue,very thin] (0,0) circle (2);
	\coordinate (x) at (225:2); 
	\coordinate (y) at (135:2);
	\draw[very thin] (0,0) -- (y);
	\draw[red,thick] (x) -- +(0.1,0.1)  (x) -- +(-0.1,-0.1) (x) -- +(0.1,-0.1) (x) -- +(-0.1,0.1);
	\draw[red,thick] (y) -- +(0.1,0.1)  (y) -- +(-0.1,-0.1) (y) -- +(0.1,-0.1) (y) -- +(-0.1,0.1);
	\draw (2,0)node[below] {$1$};
	\draw (-2,0)node[below] {$-1$};
	\draw (0,2)node[left] {$1$};
	\draw (0,-2)node[left] {$-1$};
	\draw (0,-2)node[left] {$-1$};
	\draw (0,0) (135:1cm) arc (135:180:1cm);
	\draw (-0.6,0.3)node {$\delta$};
\end{tikzpicture}
\caption{Polstellen-Nullstellendiagramm, Butterworth 2. Ordnung}
\label{fig:filter_polnul}
\end{figure}



Charakteristisch für den Butterworthfilter ist das sich die Polstellen auf einer Kreisbahn befinden. Auf die Grenzfreqeunz normiert hat dieser beim Butterworthfilter den Radius
eins. Bei einem Butterworth 2. Ordnung befinden sie sich genau bei $\delta=45^\circ$. Das Interessante am Polstellen-Nullstellendiagramm ist, dass sich Polfrequenz $\Omega_P$ und 
Polgüte $Q_P$ einfach ablesen lassen. Die Polfrequenz $\Omega_P$ ist der Betrag der normierten Polstelle, welcher beim Butterworth-Filter immer eins ist.
Die Polgüte ist abhängig von $\delta$ und ergibt sich zu: $Q_P=\frac{1}{2\cos{\delta}}$. Für unseren Butterworthfilter ergeben sich also $Q_P=0,707$ und $\Omega_P=1$


Betrachten wir deie Übertragungsfunktion eines Sallen-key Tiefpasses 2. Ordnung:

\begin{align*}
A(P)&=\frac{A_0}{1+\omega_g (R_2 C_2 + R_1 C_2 + R_1 C_2(1-A_0))P + \omega_g^2R_1 R_2 C_1C_2P^2}
\end{align*}

mit
\begin{align*}
A_0=1+\frac{R_6}{R_5}
\end{align*}


Die Bauteilwerte erhält man durch einen Koeffizientenvergleich mit der entnormierten
Übertragungsfunktion ($P=\frac{s}{\omega}$) eines Tiefpasses zweiter Ordnung:

\begin{align*}
A(P)&=\frac{A_0}{1+\frac{1}{\omega_g\Omega_PQ_P}s+\frac{1}{\omega_g^2\Omega_P^2}s^2}
\end{align*}

Die Auflösung des Vergleiches ist mit vielen Mathematischen umformungen verbunden, deswegen wird hier auf eine
externe Quelle verwiesen \cite[S. 102]{Krucker2000}.
Nach dem Koeffizientenvergleich ergibt sich

\begin{align*}
C_1&<\frac{C_2\cdot(1+4Q^2_P(A_0-1))}{4Q^2_P}\\
R_1&=\frac{1}{2\omega_g\Omega_PQ_P} \cdot \frac{C_2\pm\sqrt{C_2^2-4Q^2_PC_2(C_1+C2(1-A_0))}}{C_2(C_1-C_2(1-A_0))}   \\
R_2&=\frac{1}{2\omega_g\Omega_PQ_P} \cdot \frac{C_2\pm\sqrt{C_2^2-4Q^2_PC_2(C_1+C2(1-A_0))}}{C_1C_2}  \\
Q_p&=\frac{\sqrt{R_1R_2C_1C_2}}{C1(R_1+R_2)+R_1C_2(1-A_0)}\\
\Omega_p&=\frac{1}{\omega_g\sqrt{R_1R_2C_1C_2}}
\end{align*}

Dabei sind immernur die positiven, reellen Lösungen zu verwenden. Schließlich git es in der Realität keinen negativen Wiederstand, leider.


\subsubsection{Bestimmung der Bauteilwerte}


$Q_P=0,707$ und $\Omega_P=1$, $A_0=48$, $\omega_g = 100Hz$
$A_0$ ist die Gleichspannungsverstärkung, sie beschreibt den gewünchten Verstärkungsfaktor der Bereits in einem voherigen
Abschnitt mit 48 bestimmt wurde. Die Berechnungen wurden mit Hilfe eines Python-Scriptes ausgeführt, dabei wurden verschiedenne
Konfigurationen durchgerechnet. Hautpsächlich wurde dabei darauf geachtet, dass sich der Filter mit den vor Ort vorhandennen SMD-Bauteilen
aufgebaut werden kann.

In den Berechnungen viel auf, dass bei steigender größe der Kondensatoren die Größe der Wiederstände sinkt. Da Wiederstände auch in großen Größen vorhanden waren,
Wurde für den frei wählbaren $C_2$ ein kleiner Wert von 82nF gewählt.

\begin{align*}
C_1&<\frac{C_2\cdot(1+4\cdot0.707^2_P(48-1))}{4\cdot0.707^2}\\
C_1&<3.90\mu F
\end{align*}

$C_1$ soll nur kleiner sein als 3.90\textmu F und wird ebenfalls auf 82nF gesetzt.

\begin{align*}
R_1&=\frac{1}{2\cdot100Hz\cdot0,707} \cdot \frac{82nF\pm\sqrt{82nF^2-4\cdot0.707^2\cdot82nF(82nF+82nF(1-48))}}{82nF(82nF-82nF(1-48))}\\
R_1&=[-3176\Omega,2579\Omega]
\end{align*}


\begin{align*}
R_2&=\frac{1}{2\cdot100Hz\cdot0,707} \cdot \frac{82nF\pm\sqrt{82nF^2-4\cdot0.707^2\cdot82nF(82nF+82nF(1-48))}}{82nF^2}\\
R_2&=[146079\Omega,-118626\Omega]
\end{align*}

Da nur positive Werte genutzt werden, ergeben sich die Bauteilwerte nun zu:
\begin{align*}
C_1&=82nF\\
C_2&=82nF\\
R_1&=2579\Omega\\
R_2&=146079\Omega
\end{align*}

In der Folgenden Abbildung ist das Ergebniss der Simulation zu sehen. An der Abbildung leider nicht gut zu erkennen,
liegt der -3dB Punkt genau bei 100Hz. Die Frequenzachse des Diagrammes geht genau bis 3,9kHz. 
Es ist eine Verstärkung von 48 des Ursprungssignals gewünscht. Diese Verstärkung wird mit 33,6 dB bei 10Hz, erreicht.
\begin{align*}
20\cdot\log{48}=-33,6dB
\end{align*}

Bei 3,9kHz erreicht der Filter eine Dämpfung von -30,1dB zusammen mit der Verstärkng von 33,6dB unseres Eingangssignals,
wird das Bereits verstärkte Signal also um 63,7 dB gedämpft. Gefordert waren hier 55,9dB, so das der Filter den gerforderten wert übersteigt, wass an der niedrigeren Grenzfreqeunz von 100Hz statt 108,9Hz liegt.

%%Verstärkung: 33,6dB gewünscht:
%%20*log(1/48)=-33,6

%%dämpfung bei 3,9Khz = 30,7dB
%% 33,6+30,1 =63,7 gewüncht: 55,9
%% Grenzfreqeunz = 100Hz (-3db)
\begin{figure}[H]
\centering
\begin{gnuplot}[terminal=pdf]
  set nokey 
  set xrange [10:3900]
  set xlabel 'Frequenz in [Hz]'
  set ylabel 'Verstärkung in [dB]'
  set logscale x 10
  plot 'Simulation/Filter_original_frequenzgang.csv' with line, 'Simulation/Filter_real_frequenzgang.csv' with line
\end{gnuplot}
\caption{Frequenzgang des berechneten Filters}
\label{plott:filter_freq}
\end{figure}

Leider kann ein solcher Filter nur mit erheblichen Aufwändungen gebaut werden, da es keine fertigen Wiederstände in den Größen $2579\Omega\\$ und $146079\Omega$ gibt. Da jedoch alle Wiederstände der E12 Reihe vor Ort vorhanden sind, werden die realen Werte wiefolgt gewählt: $R_1=2,7k\Omega\\ R_2=150k\Omega$, da sie den nächsten Größen in der E12 Reihe entsprechen.


In der Folgenden Abbildung sit die Simulation des Filters mit den Realenbauteilwerten zu sehen.
Die Grenzfrequenz des Filters (-3dB) liegt diesmal mit 104Hz etwas über den ursprünglichen 100Hz. da wir die Werte von $R_5$ und $R_6$ nicht verändert haben liegt die Verstärkung bei 10Hz immernoch bei exakt 33,6dB. Bei 3,9 kHz im Diagramm gut zu erkennen wird trotz der höheren eine höhere Dämpfung als vorher erreicht. Diese liegt bei 33,7dB, daran kan mann erkennen das es sich nicht mehr um einen idealen Butterworthfilter handelt. 
%%Verstärkung: 33,6dB gewünscht:
%%20*log(1/48)=-33,6

%%dämpfung bei 3,9Khz = 30,7dB
%% 33,6+30,7 =64,3 gewüncht: 55,9
%% Grenzfreqeunz = 104Hz (-3db)
\begin{figure}[H]
\centering
\begin{gnuplot}[terminal=pdf]
  set nokey 
  set xrange [10:3900]
  set xlabel 'Frequenz in [Hz]'
  set ylabel 'Verstärkung in [dB]'
  set logscale x 10
  plot 'Simulation/Filter_real_frequenzgang.csv' with line
\end{gnuplot}
\caption{Frequenzgang des berechneten Filters mit finalen Bauteilwerten}
\label{plott:filter_freq_real}
\end{figure}


Im der folgenden Abbildung [\ref{plott:filter_sprungantwort}] ist die Antwort des Filters auf ein Rechtecksignal mit 3,9kHz, einem Tastverhältnisvon 0,5 und einer Amplitude von 50mV .Das Überschwingen im Bereich von 7ms ist charakteristisch für den Butterworthfiter und wirks sich negativ auf die Messung des Stromes aus. Allerdings werden solch große Sprünge in der Praxis nicht auftreten werden, da der Strom duch die große Induktivität des Motors nur langsam ansteigt.
\begin{figure}[H]
\centering
\begin{gnuplot}[terminal=pdf]
  set nokey 
  set yrange [0:3]
  set xlabel 'Zeit in [s]'
  set ylabel 'Spannung in [V]'
  plot 'Simulation/Filter_real_time.csv' with line
\end{gnuplot}
\caption{Sprungantwort des Filters}
\label{plott:filter_sprungantwort}
\end{figure}


Die in Abbildung [\ref{plott:ripple}] gut zu Restwellgkeit (Ripple) beträgt 3,36mV,und liegt damit deutlich unter den gerforderten 4,88mV. Als Eingangssignal dient hier ein Rechtecksignal mit 3,9kHz  und einem Tastverhältnisvon 0,5, die Amplitude liegt bei 50mV. Die Tatsache dass das Signal 240mV über den rechnerischen 2,40V  ($0.5V \cdot 48 $) liegt rührt daher dass LT-Spice die steig und fall Zeiten in den low-Bereich des Rechteck signals legt, woduruch der Mittelwert des Signals bei 2,64V liegt.
 
%% Ripple: 0,003358
%% zu hoher Spannungswert durch Steig und Fallzeiten im low bereich des Rechtecksignals
\begin{figure}[H]
\centering
\begin{gnuplot}[terminal=pdf]
  set nokey 
  set xrange [0.03:0.04]
  set yrange [2.62:2.66]
  set xlabel 'Zeit in [s]'
  set ylabel 'Spannung in [V]'
  plot 'Simulation/Filter_real_time.csv' with line
\end{gnuplot}
\caption{Restwellgkeit des Filters}
\label{plott:ripple}
\end{figure}








\bibliography{BIB}{}
\bibliographystyle{plain}

\end{document}