\chapter{Anforderungsanalyse}
\section{Anforderungen an die Platine}
Laut Regelwerk sind alle Teams zur verwendung eines elektrischen Antriebs verpflichtet.
Die Anzahl der angetriebenen Räder ist nicht vorgeschrieben
des weiteren muss das Auto durch Akkus mit Strom versorgt werden.
Die Übertragung von Daten ist während der Dauer der Disziplinen nicht gestattet

\subsection{Fahrzeugabmessungen}
Es ist ein vierrädriges Fahrzeug mit 2 Achsen im Maststab 1:10 vorzusehen. Die Spurweite, gemessen von Reifenmitte zu
Reifenmitte, muss mindestens 16 cm betragen. Der Radstand muss mindestens 20 cm betragen.
Die Höhe des Fahrzeuges darf 30 cm nicht überschreiten, über das Fahrzeug hinausragende flexible Antennen sind gestattet.
Zur Abnahme des Fahrzeuges, muss es durch ein 40 cm Breites und 30 cm Hohes Tor fahren.


\subsection{RC-Modus}
In Notsituationen muss es möglich sein das Fahrzeug mit Hilfe einer Funkfernbedienung anzuhalten und manuell zu steuern. Eine solche Notsituation tritt ein, wenn
das Auto seine Aufgabe aufgrund eines Fahrfehlers oder anderem Fehlverhalten nicht mehr autonom fortführen kann.
Der RC-Modus muss per Fernbedienung eingeschaltet und ausgeschaltet werden, bei Aktivierung des RC-Modus muss das Fahrzeug unverzüglich angehalten werden.
Während des Wettbewerbs darf die Geschwindigkeit des Autos $0,3\frac{m}{s}$ nicht überschreiten.
Da das 2,4-GHz Band bereits durch die Vorort genutzte Kameratechnik belegt ist können diese Frequenzen nicht für den RC-Modus genutzt werden.
Der RC-Modus muss durch eine blaue Leuchte an der höchsten stelle des Fahrzeuges angezeigt werden, welche mit einer Frequenz von 1-Hz blinkt.

\subsection{Signalleuchten}
Durch die Anlehnung des Wettbewerbes an den realen Straßenverkehr muss das Auto über alle in echten Auto vorhandene Signalleuchten besitzen. 
Dazu gehören 3 rote Bremslichter am Heck des Autos sowie jeweils 2 gelbe Blinker Rechts und Links am Fahrzeug.  Die Blinkfrequenz der Blinker muss
1-Hz betragen.

\subsection{Verkleidung}
Um eine schnelle Wartung und Überprüfung zu ermöglichen muss die Fahrzeug Abdeckung jederzeit schnell und einfach entfernt werden können. Des Weiterem muss die dem 
Schutzgrad IP 11 entsprechen.


\section{Resultat aus den Anforderungen}
Aufgrund der Anforderungen der elektrischen Antriebes und des Maststabes von 1:10 kann als Grundaufbau des Autos ein gängiges Ferngesteuertes Auto aus dem
freien Handel genutzt werden. Die zur Verfügung stehenden Möglichkeiten wurden dabei im Team diskutiert und die Wahl fiel dabei auf das ``TT-01R Type E'' welches
allen Anforderungen genügt. Tamiya ist seit langen ein im Modellbau etablierter Hersteller so das hier Ersatzteile lange verfügbar sein sollten.

Im Bausatz des TT-01R Type E ist bereits ein Elektromotor samt Reglerelektronik enthalten. Leider verbaut Tamiya Motoren von vielen verschiedenen Herstellern,
zu denen es leider kein Datenblatt gibt. Einige Recherchen ergaben allerdings das alle Motoren Nachbauten eines ``Mabuchi RS-540SH'' sind, von welchem ein Datenblatt
existiert \cite{Mabuchi}


