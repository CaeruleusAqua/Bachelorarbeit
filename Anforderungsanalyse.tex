
\chapter{Anforderungsanalyse}

\section{Anforderungen laut Regelwerk}
Um am „Carolo-Cup“ Teilnemen zu können ist ein Regelkonformes Fahrzeug notig, darum wird nun ein Auszug aus den Anforderungen an das Auto kurz aufgelistet und ausgewertet.
Alle Anforderungen können im Regelwerk des „Carolo-Cup“ nachgelesen werden \cite{website-carolo-cup-regelwerk}


\subsection{Fahrzeugantrieb und Energieversorgung}
Laut Regelwerk sind alle Teams zur verwendung eines elektrischen Antriebs verpflichtet.
Die Anzahl der angetriebenen Räder ist nicht vorgeschrieben
des weiteren muss das Auto durch Akkus mit Strom versorgt werden.
Die Übertragung von Daten ist während der Dauer der Disziplinen nicht gestattet

\subsection{Fahrzeugantrieb und Energieversorgung}
Es ist eine Zweiradlenkung der Vorderachse vorzusehen. Die übrige Gestaltung des Fahrwerks bleibt den Teams überlassen. Als
technische Ausprägung ist ausschließlich die Achsschenkellenkung zugelassen.

\subsection{RC-Modus}
In Notsituationen muss es möglich sein das Fahrzeug mit Hilfe einer Funkfernbedienung anzuhalten und manuell zu steuern. Eine solche Notsituation tritt ein, wenn
das Auto seine Aufgabe aufgrund eines Fahrfehlers oder anderem Fehlverhalten nicht mehr autonom fortführen kann.
Der RC-Modus muss per Fernbedienung eingeschaltet und ausgeschaltet werden, bei Aktivierung des RC-Modus muss das Fahrzeug unverzüglich angehalten werden.
Während des Wettbewerbs darf die Geschwindigkeit des Autos $0,3\frac{m}{s}$ nicht überschreiten.
Da das 2,4-GHz Band bereits durch die Vorort genutzte Kameratechnik belegt ist können diese Frequenzen nicht für den RC-Modus genutzt werden.
Der RC-Modus muss durch eine blaue Leuchte an der höchsten stelle des Fahrzeuges angezeigt werden, welche mit einer Frequenz von 1-Hz blinkt.

\subsection{Signalleuchten}
Durch die Anlehnung des Wettbewerbes an den realen Straßenverkehr muss das Auto über alle in echten Auto vorhandene Signalleuchten besitzen. 
Dazu gehören 3 rote Bremslichter am Heck des Autos sowie jeweils 2 gelbe Blinker Rechts und Links am Fahrzeug.  Die Blinkfrequenz der Blinker muss
1-Hz betragen.

\section{Anforderungen durch Bewertungskriterien}
Weiter Anorderungen ergeben sich aus den Bewertungskriterien. Während dern statischen Disziplinen muss das Team ihr Gesamtkonzept präsentieren. Schwerpunkte dabei sind,
Hardware und Software Architektur sowie Energiebedarf und Herstellungskosten \cite{website-carolo-cup-regelwerk}. Daraus entstehen weitere Anforderungen: Energieeffizienz
und Herstellungskosten. 

In den dynamischen Disziplinen soll das Fahrzeug eine Stecke völlig autonom abfahren. Solch eine wie in Abbildung [\ref{fig:Rundkurs}] ausehen.Das Fahzeug darf seine
Fahrspur dabei nie mit mehr als einem Rad verlassen. Um die Fahrbahn zu erkenn wird vom Team eine auf dem Auto montierte Kamera verwendet. Da laut Regelwerk keine
Daten vom oder zum Auto übertragen werden dürfen, muss die Verarbeitung der Bilder auf dem Auto stattfinden. Daher ist auf dem Auto eine leistungsfähige Recheneinheit
von nöten. Hierzu wurden vom Lehrstuhl boards des Types ``Pandaboard ES'' zur verfügung gestellt. 

Innerhalb des parallelen Einparkens ist es nötig die Größe der Parklücke zu erkennen, bzw Abstände zu Objekten zumessen.

\section{Anforderungen durch Teammitglieder}

Inertialsensorik..


\section{Auswertung der Anforderungen}
Da ein elektrischer Antrieb vorgesehen ist, muss die nötige Elektronik zur Steuerung des Motors in die Platine integriert werden.
Der Einfachheit halber wird die Lenkung von einem Servomotor übernommen.
Damit das Auto die im RC-Modus nötigen Funksignale auswerten kann muss ein Empfänger integriert werden.
Desweiteren muss das Auto mit den nötigen Signalleuchten ausgestattet sein, dazu gehören Blinker rechts und links jeweils vorne und hinten.
Sowie 3 Bremsleuchten an der Rückseite und eine weitere Leuchte welche din RC-Modus anzeigt. Der vollständigkeit halber kommt hier noch die
Frontbeleuchtung hinzu

Zum bestimmen von Entferneungen zu andern Autos oder Hindernissen werden Distanzsensoren benötigt

Da Energiebedarf eine Anforderung ist sollte bei der Auswahl der Komponenten auf Energieeffizienz geachtet werden.

Zusammenfassend hier eine Liste mit den benötigten Komponenten.

\begin{itemize}
 \item Elektonik zur Motoransteuerung
 \item Servomotor zur Lenkung
 \item Funk Empfänger für den RC-Modus
 \item Beleuchtung
 \item Distanzsensoren
 \item Interialsensorik
\end{itemize}







% Das Team setzt ein Fahrzeug vom Typ TT-01 Type E des Herstellers Tamiya ein. Dabei handelst es sich um ein modellbau Auto. Dieses Modell wird bereits mit
% Motor und Fahrtenregler ausgeliefertund entspricht den vorgaben im Regelwerk. Da am Lehrstuhl bereits größere Mengen an LiPo-Akkus mit einer Nennspannung von 14,4V vorhanden sind, sollen diese verwendent
% werden. Da der dem Auto beiliegende Fahrtenregler nicht für eine derart hohe Spannung ausgelegt ist, ist hier ein Ersatz notwendig. 
% 

% 
% Im folgenden wird geklärt wie das Fahrzeug vom Team geplant und umgestetzt wird. Das Fahrzeug basiert auf einem Modellbauauto vom Typ TT-01 Type E des Herstellers Tamiya. 
% Das Auto besteht dabei aus folgenden Komponenten: einem Elektromotor zur Fortbewegung, einem Servomotor zur Lenkung, einer Kamera welche die Fahrbahn aufzeichnet, einer Recheneinheit welche
% die Bildverabreitung und Regelung des Fahrzeuges übernimmt und einer Treiberplatine. 
% 
% 
% \subsection{Die Treiberplatine}
% Die Treiberplatine hat hat die Aufgabe die Sensorik und Aktorik mit der Recheneinheit zu verbinden. Zusätzlich befinden sich auf ihr eine Spannungsversorgung, ein Motortrieber und
% diverste Anschlüsse fürdie Sensorik. Die Platine wurde in 2 Versionen erstellt. Da für die 1. Platinenversion eine Ansteuerung durch GPIO-Pins des Pandaboards geplant war. Durch den
% späteren Wechsel zum Intel NUC, welcher nicht über GPIO-Pins verfügt, musste eine überarbeite Version der Platine entwickelt werden. Welche ien einfachen Anschluss an den NUC gewährleistet.
% Die 
% 
% Die Treiberplatine wurde 
% 
% In Abbildung [\ref{fig:platine_rv3}] ist das Board der überarbeiteten Platinenversion zu sehen.
% 
% \begin{figure}[H]
% \centering
% \includegraphics[width=.8\textwidth]{platine_rv3.png}\\
% \caption{Korrigierte Platinenversion}
% \label{fig:platine_rv3}
% \end{figure}
% 
% 
% 
% 
% 
% 
% 
% 












